%===============================================
%   rhnotes.tex -- Framework for RH-coding Notes
%===============================================
\documentclass[11pt,a4paper]{article}

%----- ENHANCED TYPOGRAPHY -----
\usepackage[utf8]{inputenc}
\usepackage[T1]{fontenc}
\usepackage{lmodern}        % clean vector font
\usepackage{microtype}      % better justification & kerning
\usepackage{palatino} 
\usepackage{braket}    % Palatino for text & math

%----- PAGE LAYOUT -----
\usepackage{geometry}
\geometry{top=1in, bottom=1in, left=1in, right=1in}
\usepackage{setspace}
\onehalfspacing  % 1.5 line spacing

%----- FANCY HEADERS & FOOTERS -----
\usepackage{fancyhdr}
\pagestyle{fancy}
\fancyhf{}
% page number outside, header text inside
\fancyhead[LE]{\small Georges Khater}
\fancyhead[RE]{\small Notes on RH-coding}
\fancyhead[LO]{\small \rightmark}
\fancyhead[RO]{\small \leftmark}
\renewcommand{\headrulewidth}{0.4pt}
\renewcommand{\footrulewidth}{0pt}

% make sections feed into \leftmark/\rightmark
\renewcommand{\sectionmark}[1]{\markboth{#1}{}}
\renewcommand{\subsectionmark}[1]{\markright{#1}}

%----- SECTION NUMBERING & TOC DEPTH -----
\setcounter{secnumdepth}{3}  % number down to \subsubsection
\setcounter{tocdepth}{2}     % show ToC down to \subsection

%----- AMS MATH & THEOREM STYLES -----
\usepackage{amsmath,amssymb,mathtools}
\usepackage{amsthm}

% definitions, examples, remarks upright
\theoremstyle{definition}
\newtheorem{definition}{Definition}[section]
\newtheorem{example}[definition]{Example}
\newtheorem{remark}[definition]{Remark}

% theorems, lemmas, corollaries italic
\theoremstyle{plain}
\newtheorem{theorem}[definition]{Theorem}
\newtheorem{lemma}[definition]{Lemma}
\newtheorem{proposition}[definition]{Proposition}
\newtheorem{corollary}[definition]{Corollary}

% unnumbered proof environment
\theoremstyle{remark}

%----- OTHER PACKAGES -----
\usepackage{graphicx}
\usepackage{tikz}
\usetikzlibrary{calc, matrix, decorations.pathreplacing, positioning}
\usepackage{tikz-cd}
\usepackage{hyperref}
\hypersetup{colorlinks,
linkcolor=blue, citecolor=purple, urlcolor=teal}
\usepackage{enumitem}
\setlist[itemize]{nosep, left=1.5em}
\usepackage{booktabs}
\usepackage{listings}
\lstset{
basicstyle=\ttfamily\small,
numbers=left,
numbersep=5pt,
frame=single,
breaklines=true
}
\usepackage{xcolor}
\definecolor{shade}{HTML}{F5F5F5}
\usepackage{float}
%----- CUSTOM MACROS -----
\newcommand{\F}{\mathbb{F}}
\newcommand{\code}[1]{\texttt{#1}}
\newcommand{\bsc}{\mathrm{BSC}}
\newcommand{\dist}[2]{d\bigl(#1,#2\bigr)}
\newcommand{\R}{\mathbb{R}}
\newcommand{\Z}{\mathbb{Z}}
\newcommand{\N}{\mathbb{N}} 
\newcommand{\Q}{\mathbb{Q}} 
\renewcommand{\set}[1]{\left\{ #1 \right\}}
\newcommand{\angles}[1]{\langle #1 \rangle}
\newcommand{\abs}[1]{\lvert #1 \rvert}
\newcommand{\norm}[1]{\lVert #1 \rVert} 

% \usepackage{mathtools} 
% \DeclarePairedDelimiter{\angles}{\langle}{\rangle} 
% \DeclarePairedDelimiter{\braces}{\left\{}{\right\}} 
% \DeclarePairedDelimiter{\abs}{\lvert}{\rvert} 
% \DeclarePairedDelimiter{\norm}{\lVert}{\rVert}

%----- TITLE METADATA -----
\title{\LARGE\bfseries Topology I}
\author{Georges Khater \\ \small American University of Beirut, Math 214}
\date{\today}

%===============================================
\begin{document}
\maketitle
\tableofcontents
\bigskip

\section{Introduction}
These notes were taken as part of my course in topology at AUB.
They were written primarily for personal use rather than publication, and should therefore not be regarded as a complete or polished textbook treatment; especially in the earlier, more elementary sections.
Nevertheless, they are intended to provide a rigorous and coherent introduction to the material.

Some routine arguments and proofs that I was already familiar with have been omitted.
These notes are meant to be read alongside a standard reference text, which provides the full formal development.
As in any mathematics course, the notes represent only half of the learning experience, the other half consisting of the problem sets.
For academic integrity reasons, I cannot include my solutions here.

Since these notes were taken during lectures, occasional typos or inconsistencies are inevitable.
Feel free to contact me if you notice any.

\subsection{Motivating examples} 
\begin{theorem} [Brouwer's fixed point theorem]
    If we have an $n$-disk (closed disk) $D^n$ and a continuous function 
    $$f \colon D^n \to D^n$$
    Then $f$ has a fixed point (a point such that $f(x) = x$). 
\end{theorem}
Note: The shape here is not important, since the closed disk is equivalent to a closed rectangle for example. However, if we replace the disk by 
a disk with a hole in the middle, then the example doesn't hold anymore; consider the map that just rotates by a non-zero angle. 
Similarly, it is also important that the disk is closed (it includes its boundary). \\
{\tiny \textit{There is a cute combinatorial proof of this theorem, which uses Sperner's lemma.}}

\paragraph{Euler's identity}
Consider a planar connected graph (edges don't intersect, graph is on a plane and is connected), and let $V, E, F$ be the number of vertices, faces, edges 
respectively. Then 
$$V - E + F = 2$$
Note that if we draw the graph on a sphere instead of a plane, this still holds, but if we draw on a Torus, we get a different constant, view introduction drawing on Euler's identity. 

\paragraph{Euclidean spaces} Here let $X = \mathbb{R}^d$, $d \in \{1, 2, 3, \cdots\}$. 
$x_1, x_2, \cdots$ and $x_0$ points of $X$. 
we say that $x_1, x_2, \cdots$ converges to $x$ iff 
$$\forall \varepsilon > 0, \, \exists n_0, \quad \forall n \geq n_0, \ d(x_n, x) < \varepsilon$$
i.e $x_n \in B_\varepsilon (x)$. 
Equivalently, for every open set $x \in U$, $\exists n_0$ $\forall n \geq n_0, \ x_n \in U$. 

Let $f \colon \R^d \to \R^d$, we say that $f$ is continuous iff $\forall \varepsilon > 0$ $\exists \delta > 0$ s.t 
$\forall x, x' \in \R^d$ we have 
$$d(x, x') < \delta \implies d(f(x), f(x')) < \varepsilon$$
i.e $f \left(B_\delta (x) \right) \subseteq B_\varepsilon (f(x))$. Equivalently, using sequences: 
$$x_n \to x_0 \implies f(x_n) \to f(x_0)$$

A set $U \subseteq \R^d$ is open iff 
$$\forall x \in U \ \exists \varepsilon, \, B_\varepsilon(x) \subseteq U$$
Conversely, a set is closed iff its complement is open. The closure of a set $A$ (denoted $\overline{A}$) is defined by 
$$ A \to \overline{A} = \{x \in \R^d \colon \: \forall \varepsilon > 0, \, B_\varepsilon(x) \cap A \neq \emptyset\}$$ 
The interior of the set $\mathring{A}$ is 
$$A \to \mathring{A} = \{x \in \R^d \colon \: \exists \varepsilon, \, B_\varepsilon(x) \subseteq A\}$$

\subsection{Generalization of Topology}
A Topology is a set with a structure that allows us to discuss these kinds of concepts; We have a choice as in what to generalize (continuity, convergence, open balls / neighborhoods). 
All of these were tried, some structures matched, some were more general than others. The popular modern approach is to generalize the concept of \emph{open sets}, i.e a topology can 
be described by specifying which of its subsets are open, we call the set of all open sets $\mathcal{T} \subseteq \mathcal{P}(X)$. 

\begin{definition}[Topological Space]
    A \emph{Topological space} $(X, \mathcal{T})$ is a set $X$ equipped with a topology (family of open sets) $\mathcal{T} \subseteq \mathcal{P} (X)$ which 
    respects the following axioms: 
    \begin{itemize}
        \item The union of any (arbitrary) collection of open sets is again open 
        $$\forall S \subseteq \mathcal{T} \implies \bigcup_{A \in S} A \in \mathcal{T}$$
        
        \item The intersection of a finite collection of open sets is again open 
        $$A_1, \cdots, A_n \subseteq \mathcal{T} \implies A_1 \cap \cdots \cap A_n \in \mathcal{T}$$

        \item $\emptyset, X \in \mathcal{T}$
    \end{itemize}
    We say that a set $U \subseteq X$ is closed iff its complement is open.
\end{definition}

\begin{example}[Trivial and discrete Topologies]
    Let $X \neq \emptyset$ be an arbitrary open set, let $\mathcal{T} = \{\emptyset, X\}$. Then 
    $(X, \mathcal{T})$ is a valid topology (called the \emph{trivial topology}).  

    Let $X \neq \emptyset$, and let $\mathcal{T} = \mathcal{P} (X)$. Then $(X, \mathcal{T})$ 
    is a valid topology (called the \emph{discrete topology}). 
\end{example}

\begin{definition}[Convergence]
    A sequence $x_1, x_2, \cdots$ in $X$ converges to a point $x \in X$ iff for every open set 
    $U \ni x$, there exists $n_0$ s.t 
    $$\forall n \geq n_0, \, x_n \in U$$
\end{definition}

\begin{definition}[Continuity]
    Let $(X, \mathcal{T}), (Y, \mathcal{S})$ be topological spaces, then a function (mapping) 
    $$f \colon \: X \to Y$$
    is said to be continuous iff 
    $$\forall U \in \mathcal{S}, \, f^{-1}(U) \in \mathcal{T}$$ 
\end{definition}

Notice that for example in the trivial topology, every sequence converges to every point; in particular in this 
topology the limit of sequences is not unique. Similarly any $f$ that maps to a trivial topology must be continuous. \\
Conversely, any $f$ maps from a discrete topology must be continuous. Moreover, convergence in this topology that the 
sequence is eventually constant. \\
Less obviously, Say we have $f \colon \R^d \to X$ (where $X$ is equipped with the trivial topology) then $f$ is constant 
because the only clopen sets are the emptyset and $X$. 

\begin{example}
    Let $X = \{0,1\}$ and $\mathcal{T} = \{\emptyset, \{0\}, \{0,1\}\}$ is a valid topology. 
\end{example}

\begin{example}[The cofinite Topology]
    Let $X \neq \emptyset$ be an arbitrary set, then $A \subseteq X$ is declared to be open iff 
    \begin{itemize}
        \item $X \setminus A$ is finite 
        \item $A = \emptyset$
    \end{itemize}  
    This is a valid topology called the \emph{co-finite topology}. 

    Here convergence means that for every finite set $U$ that doesn't contain $x$, $\exists n_0$ such that $\forall n \geq n_0$ 
    $x_n \not \in U$. 

    Continuity of a function $f \colon X \to X$ means that for every finite set $U$, its complement is finite. 
\end{example}

\subsection{Topological Basis} 
This imitates the concept of open Balls in $\R^d$ (a set is open iff for every point in that set there is an open ball around 
this point contained in this set). 

\begin{definition}[Topological Basis]
    A \emph{Topological Basis} on a set $X$ is a family $\mathcal{B} \subseteq 2^X$ s.t 
    \begin{enumerate}
        \item $\bigcup_{A \in \mathcal{B}} A = X$. 
        \item If $x \in B_1 \cap B_2$, with $B_1, B_2 \in \mathcal{B}$, then there exists 
        $B_3 \subseteq B_1 \cap B_2$ s.t $x \in B_3$. 
    \end{enumerate}
    The topology generated by a basis $\mathcal{B}$ is 
    \begin{align*}
        \mathcal{T} &= \mathcal{T (B)} \\
        &= \left\{\bigcup_{A \in S} \colon \: S \subseteq \mathcal{B}\right\} \\
        &= \{U \subseteq X \colon \: \forall x \in U, \, \exists B \in \mathcal{B} \text{ s.t } x \in A \subseteq U\}
    \end{align*}
\end{definition}


\begin{definition}
    The topology \emph{generated by} $\mathcal{B} \subseteq 2^X$ denoted 
    \begin{align*}
        \mathcal{T} &= \mathcal{T}(\mathcal{B}) = \set{\bigcup_{A \in S} \colon S \subseteq \mathcal{B}} \\
        &= \set{U \subseteq \mathcal{X} \colon \forall x \in U, \, \exists B \in \mathcal{B} \text{ s.t } x \in B \subseteq U}
    \end{align*}
    Note that these two definitions are equivalent. 
\end{definition}

\begin{proposition}
    If $B$ is a basis, then $\mathcal{T}(\mathcal{B})$ is a topology. 
\end{proposition}

\begin{example}
    $X = \R^2$ where $\mathcal{B}_1 = \set{U \colon U \text{ is an open ball}}$, $\mathcal{T}_1 = \mathcal{T}(\mathcal{B}_1)$. We could also pick 
    $\mathcal{B}_2 = \set{R \colon R \text{ is an open rectangle}}$, with $\mathcal{T}_2 = \mathcal{T}(\mathcal{B}_2)$. 
    Look at the drawing 1 in Topological basis, then $\mathcal{T}_1 = \mathcal{T}_2$. This is because we can always draw an open rectangle within an open ball 
    and vice versa.  
\end{example}

\begin{proposition}
    Let $\mathcal{B}_1$ and $\mathcal{B}_2$ be two bases on a set $X$. Then 
    $$\mathcal{T}(\mathcal{B}_1) \subseteq \mathcal{T}(\mathcal{B}_2)$$ 
    if and only if $\forall x \in X$ and $\forall B_1 \in \mathcal{B}_1$ with $x \in B_1$, 
    $$\exists B_2 \in \mathcal{B}_2 \text{ s.t } x \in B_2 \subseteq B_1$$

    In this case we say that $\mathcal{T}_2$ is \emph{finer} than $\mathcal{T}_1$. Similarly we say that 
    $\mathcal{T}_1$ is \emph{coarser} than $\mathcal{T}_2$. 
\end{proposition}

\begin{proposition}
    Let $\mathcal{B} \subseteq 2^X$ be an arbitrary family. 
    $$\mathcal{T} = \set{\bigcup_{A \in \mathcal{A}} A \colon \ \mathcal{A} \subseteq \mathcal{B}}$$
    Then $\mathcal{T}$ is a topology iff $\mathcal{B}$ is a basis. 

    Note: this is a characteristic property of bases. 
\end{proposition}

\begin{example}[Some topologies on $\R$]
    \begin{itemize}
        \item The family of open intervals generate the Euclidean topology on $\R$. 
        (In general the family of open $n$-disks generate the Euclidean topology on $\R^n$). 

        \item Let $X = \R$, $\mathcal{B}_l = \set{[a, b) \colon \: a, b \in \R}$ (which is a valid basis). This basis generates what we call 
        the lower limit topology, the concept of convergence is unique and interesting here ($\varepsilon-\delta$ convergence from the right). 

        \item Let $X = \R$, $\mathcal{B}_k = \set{(a, b) \colon \: a, b \in \R} \cup \set{(a, b) \setminus K \colon a, b \in \R}$ where $K = \set{1/n \colon \: n \in \Z^+}$. 
    \end{itemize}
\end{example}

\begin{example}[Furstenberg Topology]
    Let $X = \Z$, $\mathcal{B} = \set{\text{collection of all arithmetic progressions}}$. i.e the sets $\angles{n,k}$ of elements $n + k \cdot d$ for all $k \in \Z$, in other words 
    for some $a, n \in Z$, $n \neq 0$ we have  
    $$\mathcal{B}(a, n) = a + n \Z = \set{a + jn \colon j \in \Z}$$
\end{example}

\begin{example}
    $X = \set{0,1}^{\Z^+}$, a point in $X$ is $x = x_1 x_2 \cdots$ (an infinite sequence of $0$s and $1$s),
    we call these points \emph{configurations} of $0$s and $1$s. 

    A \emph{word} of length $n$ is a finite sequence $w = w_1 w_2 \cdots w_n$, we allow $n = 0$ in which case we denote it by $\lambda$. 
    We define the \emph{cylinder} of base $w$ to be
    $$C(w) = \set{x \in X \colon \: x_1 \cdots x_n = w_1 \cdots w_n}$$ 
    Note that $C(\lambda) = X$. 

    Verify that the family of all cylinders form a basis on $X$. 
\end{example}

\begin{definition}
    Let $X$ be a set: 
    \begin{itemize}
        \item A \emph{partial order} on $X$ is a relation $\sqsubseteq$ which satisfies 
        \begin{enumerate}
            \item Reflexivity: $\forall x \in X$ we have 
            $$x \sqsubseteq x$$
            
            \item Antisymmetry: If $ a \sqsubseteq b$ and $b \sqsubseteq a$ then 
            $$a = b$$

            \item Transitiviy: If $a \sqsubseteq b$ and $b \sqsubseteq c$ then 
            $$a \sqsubseteq c$$
        \end{enumerate}

        \item A \emph{Total order} on $X$ is a relation $\leq$ is a partial order in which every two elements are comparable, i.e 
        $\forall a, b \in X$ we have that 
        $$a \leq b \text{ or } b \leq a$$
    \end{itemize}
\end{definition}


\begin{example}[Order Topology]
   \begin{itemize}
        \item  $X = \R^2$ and $\mathcal{B} \subseteq 2^X$ consists of all open intervals $(a, b) = \set{x \in X \colon \: a \sqsubset x \sqsubset b}$ where $a, b \in \R^2$
        according to lexicographic order. Look at the drawing. Think about what convergence and continuity mean in this topology.
    
        \item More generally, let $(X, \sqsubseteq)$ be a totally ordered set, we can define a basis 
        $$\mathcal{B} = \set{(a, b) \colon a, b \in X} \cup \set{[m, b) \colon m, b \in X, \ m = \min X} \cup \set{(a, M) \colon a, M \in X, \ M = \max X}$$
        With 
        $$(a, b) := \set{a \sqsubset x \sqsubset b}$$
        
        \item If we have a paritally ordered set, instead of look at the minimal / maximal elements instead of the minimum / maximum of $X$, note we would 
        have to add to the basis (union) the set 
        $$\set{[m , M] \colon \: m \text{ minimal }, \, M \text{ maximal}}$$ 
   \end{itemize}
\end{example}

\paragraph{Different Topologies on $\R^d$} 

View the picture 
\begin{itemize}
    \item \textbf{Discrete topology: } This is the finest possible topology. 
    \item \textbf{Trivial topology: } This is the coarsest possible topology. 
    \item \textbf{Euclidean topology} 
    \item \textbf{Co-finite topology}
\end{itemize}

\begin{remark}
    Let $\set{\mathcal{T}_i}_{i \in I}$ be a collection of topologies on a set $X$. Then 
    $$\bigcap_{i \in I} \mathcal{T}_i \quad \text{is a valid topology}$$
    Indeed, $\varnothing, X \in \mathcal{T}_i$ for all $i$, and arbitrary unions / finite intersections of sets that lie in every $\mathcal{T}_i$ still lie in every $\mathcal{T}_i$; hence they lie in $\bigcap_{i \in I} \mathcal{T}_i$. 
\end{remark}

\subsection{Topology generated by a set}
Let $\mathcal{B} \subseteq 2^X$ be a basis, then 
    $$ \mathcal{B} \to \mathcal{T} = \set{\bigcup_{A \in \mathcal{A}} A \colon \ \mathcal{A} \subseteq \mathcal{B}}$$
We wish to generalize this to an arbitrary set $\mathcal{S}$. i.e we want to define 
    $$\mathcal{S} \subseteq 2^X \to \mathcal{T}(\mathcal{S}) = \bigcap_{\substack{\hat{\mathcal{T}}\ \text{a topology on } X\\ \mathcal{S} \subseteq \hat{\mathcal{T}}}} \hat{\mathcal{T}}$$
We know that the intersection is nonempty since we have at least the discrete topology. 

\begin{lemma}
    If $\mathcal{B} \subseteq 2^X$, then the two definitions of $\mathcal{T}(\mathcal{B})$ coincide. 
\end{lemma}

\begin{proof}
    Let $\mathcal{T}_1 = \set{\bigcup_{A \in \mathcal{A}} A \colon \ \mathcal{A} \subseteq \mathcal{B}}$, and let $\mathcal{T}_2$ be the coarsest topology containing $\mathcal{B}$. 
    \begin{itemize}
        \item[($\mathcal{T}_1 \supseteq \mathcal{T}_2$)] This is clear because $\mathcal{T}_2$ is the coarsest topology 
        containing $\mathcal{B}$ and $\mathcal{T}_1$ contains $\mathcal{B}$.

        \item[($\mathcal{T}_1 \subseteq \mathcal{T}_2$)] Each element $\mathcal{O}$ of $\mathcal{T}_1$ is of the form 
        $$\mathcal{O} = \bigcup_{A \in \mathcal{A}} A \quad \text{for some } \mathcal{A} \subseteq \mathcal{B}.$$
        But $\mathcal{B} \subseteq \mathcal{T}_2$, hence $\mathcal{A} \subseteq \mathcal{T}_2$, therefore $\mathcal{O}$ is a union of 
        open sets of $\mathcal{T}_2$ and hence it is open in $\mathcal{T}_2$. 
    \end{itemize}
\end{proof}

\begin{lemma}
    Let $\mathcal{S} \subseteq 2^X$ be arbitrary, then 
    $$\mathcal{B}(\mathcal{S}) := \set{A_1 \cap \cdots \cap A_n \mid \ n \geq 0, \ A_1, \cdots, A_n \in \mathcal{S}}$$
    is a basis. 
\end{lemma}

\begin{proposition}[Explicit characterisation of $\mathcal{T}(\mathcal{S})$]
    \mbox{}\\
    Let $\mathcal{S} \subseteq 2^X$ be arbitrary, then: 
    \begin{enumerate}[label=(\roman*)]
        \item $\mathcal{T}(\mathcal{S}) = \mathcal{T}(\mathcal{B}(\mathcal{S}))$. 
        \item In particular, (and more explicitly): 
        $$\mathcal{T}(\mathcal{S}) = \set{\text{all unions of finite intersections of elements of $\mathcal{S}$}}$$
    \end{enumerate}    
\end{proposition}

\begin{definition}
    A family of $\mathcal{S} \subseteq 2^X$ is called a \emph{subbasis of $X$} iff 
    $$\bigcup_{A \in S} A = X$$
\end{definition}

\subsection{Product Topologies} 
Let $(X, \tau_X), (Y, \tau_Y)$ be two topological spaces, then 
$$\mathcal{B} = \set{U \times V \colon U \in \tau_X, \ V \in \tau_Y}$$
is a basis on the product on $X \times Y$.     

\begin{definition}
    The \emph{product topology} or \emph{box topology} is the topology generated by $\mathcal{B}$ defined above. \\
    Note for later that box and product topologies do not coincide for an infinite product of spaces; they do coincide for finite products. 
    for finite product of spaces. 
\end{definition}

\begin{proposition}[A simpler basis]
    $B_X$ is a basis for $(X, \tau_X)$ and $B_Y$ a basis for $(Y, \tau_Y)$. Then 
    $$B_{X \times Y} := \set{A \times B \colon A \in B_X, \, B \in B_Y}$$ 
    is a basis for the product topology on $X \times Y$.  
\end{proposition}

\begin{remark}
    $$S := \set{U \times Y \colon \: U \in \tau_X} \cup \set{X \times V \colon \: V \in \tau_Y}$$
    is a sub-basis for the product topology. 
\end{remark}

\subsection{Convergence and Continuity in terms of Bases and Sub-bases}
\paragraph{Convergence: } Let $(X, \tau)$ be a topological space. 
\begin{enumerate}
    \item Suppose $B$ is a basis for $\tau$; then 
    $$x_1, x_2, \cdots \to x \text{ in } \tau \iff \forall A \in B \ \exists n_0 \ \forall n \geq n_0 \ x_n \in A$$
    i.e we only need to check for elements of the basis containing $x$. 

    \item Suppose $S$ is a sub-basis for $\tau$; then 
    $$x_1, x_2, \cdots \to x \text{ in } \tau \iff \forall A \in S \ \exists n_0 \ \forall n \geq n_0 \ x_n \in A$$
    This is true by taking a finite intersection of such $A$'s and taking the maximum $n$ over all these sets; then use (1). 
\end{enumerate} 

\paragraph{Continuity: } Let $(X, \tau_X)$, $(Y, \tau_Y)$ be topological spaces and $f \colon X \to Y$ a map: 
\begin{enumerate}
    \item Suppose $B_Y$ is a basis for $\tau_Y$. Then $f$ is continuous iff 
    $$\forall B \in B_Y, \ f^{-1}(B) \in \tau_X$$
    (This comes from the fact that preimage behaves nicely with union)
    
    \item Similarly, let $S_Y$ be a subbasis for $\tau_Y$, then $f$ is continuous iff 
    $$\forall S \in S_y, \ f^{-1}(S) \in \tau_X$$
    (This comes from the fact that preimages are behave nicely with intersections)
\end{enumerate} 

\subsection{Topological Subspaces}  
Sometimes we want to view subsets of our topological spaces as spaces of their own, in a way such that they "inherit" the original space's construction. 
Think of $[a, b] \subseteq \R$, $\Q \subset \R$, $[a, b] \times [c, d] \subset \R^2$ or even $S^d \subset \R^{d+1}$ where 
$$S^d = \set{x \in \R^{d+1} \mid \norm{x } = 1} \quad \text{ is the $d$ dimensional sphere}$$

\begin{definition}
    Let $(X, \tau)$ be a topological space, and $Y \subseteq X$. Then we define the \emph{subspace topology} or 
    \emph{relative topology} $\tau_Y$ on $Y$ where 
    $$\tau_Y = \set{A \cap Y \colon \ A \in \tau}$$
\end{definition}

\begin{proposition}
    Let $X, Y$ be topological spaces and $X' \subseteq X$ and $Y' \subseteq Y$, then the following two topologies on $X' \times Y'$ are equivalent: 
    \begin{itemize}
        \item The product topology $X' \times Y'$ where $X'$ and $Y'$ are given the subspace topology. 
        \item The subspace topology inherited from $X \times Y$. 
    \end{itemize} 
\end{proposition}

\begin{example}
    The proposition above says that in the case of $[a, b] \times [c, d]$ it doesn't matter if we take it as a subspace of $\R^2$ or as the product of two 
    subspaces of $\R$. 
\end{example}

\begin{example}[Basis for a subspace] 
    \begin{itemize}
        \item Let $X = \R$ and $B$ be the basis of open intervals of $\R$; let $Y = [a, b]$. Then we get a natural basis 
        $$B_Y = \set{(c, d) \mid \ a \leq c < d \leq b} \cup \set{[a, d) \mid \ a < c \leq b} \cup \set{(c, b] \mid a \leq c < b} \cup \set{[a, b]}$$
        
        \item If $X = \R^2$ and $B$ is the set of open balls, then $B_Y$ is the set of all open arcs. 
    \end{itemize}
\end{example}

\subsection{Closure, Interior, Boundary, Limits, etc.}

\begin{definition}[Neighborhoods]
    \begin{itemize}
        \item An \emph{open neighborhood} of $X$ in a space $X$ is an open set $U \ni x$.
        \item A \emph{neighborhood} of $X$ is a set $V$ that contains an open neighborhood of $x$ (so $x \in \mathring{V}$). 
    \end{itemize}
\end{definition}

\begin{definition}[Closure of a set]
    Let $E \subseteq X$, then we define the \emph{closure} of $E$ to be
    $$\operatorname{cl} (E) = \overline{E} := \set{x \in X \mid \ \forall U \ni x \text{ open }, \ E \cap U \neq \emptyset}$$
    Equivalently, 
    $$\overline{E} := \bigcap_{C \text{ closed s.t } E \subseteq C} C$$
    (Exercise: show the equivalence; use complements / contrapositives.)
\end{definition}

\begin{definition}[Interior of a set]
    Let $E \subseteq X$, then we define the \emph{interior} of $E$ to be 
    $$\mathring{E} := \set{x \in X \colon \: \exists N \ni x \text{ open s.t } N \subseteq E}$$
    Equivalently, 
    $$\mathring{E} := \bigcup_{B \subseteq E \text{ s.t } B \text{ open}} B$$
\end{definition}

\paragraph{Some basic properties} 
\begin{enumerate}[label = (\roman*)]
    \item $\mathring{E} \subseteq E \subseteq \overline{E}$ 
    \item $\overline{(\overline{E})} = \overline{E}$
    \item $A \subseteq B \implies \overline{A} \subseteq \overline{B}$
    \item $\overline{A \cup B}$ = $\overline{A} \cup \overline{B}$ 
    \item $\overline{\emptyset} = \emptyset$ 
    \item $E$ closed $\iff \overline{E} = E$
    \item $X \setminus \mathring{E} = \overline{(X \setminus E)}$
    \item $X \setminus \overline{E} = \operatorname{int} (X \setminus E)$
\end{enumerate}
Try to prove these, $(iv)$ is the most subtle, note that this isn't true for the complement, take $A = \Q$, $B = \R$, also note that this isn't true 
for the infinite case; consider the singletons $\set{1/n}_{n \in \N}$. 

\begin{definition}[Boundary of a set]
    Let $E \subseteq X$, then the \emph{boundary} of $E$ is 
    $$\partial E := \overline{E} \setminus \mathring{E} = \overline{E} \cap \overline{(X \setminus E)}$$
    Equivalently: 
    $$\partial E = \set{x \in X \colon \ \forall N \ni x \text{ open},\ N \cap E \neq \emptyset \text{ and } N \cap E^c \neq \emptyset}$$
\end{definition}

\begin{proposition}
    \begin{itemize}
        \item $\mathring{E} \cap \partial E = \emptyset$ 
        \item $\partial E = \emptyset \iff E \text{ is clopen}$
        \item $\overline{E} = \mathring{E} \cup \partial E$ 
        \item $\partial (\partial E) \subseteq \partial E$ 
        \item $\partial E = \partial (X \setminus E)$ 
    \end{itemize}
\end{proposition}

\begin{definition}
    \begin{itemize}
        \item A point $x \in E$ is said to be an \emph{isolated point of $E$} iff there exists a neighborhood of $N \ni x$ s.t 
        $$N \cap E = \set{x}$$
        We note 
        $$\operatorname{isolated} (E): \text{ set of isolated points of $E$}$$

        \item $x \in X$ is a l\emph{limit point of $E$} iff $\forall N \ni x$ 
        $$(N \setminus \set{x}) \cap E \neq \emptyset$$
        We note 
        $$\operatorname{lim}(E): \text{ set of limit points of $E$}$$
    \end{itemize}
\end{definition}

\begin{proposition}
    \begin{align*}
        \overline{E} &= \operatorname{Lim}(E) \sqcup \operatorname{isolated} (E) \\
        &= \operatorname{Lim} (E) \cup E
    \end{align*}
\end{proposition}

\begin{proposition}
    $x$ is the limit of a sequence in $E$, then $x \in \overline{E}$. \\
    Likewise, $x$ is the limit of a sequence in $E \setminus \set{x} \implies x \in \operatorname{lim} (E)$. 
\end{proposition}

\begin{example}
    Let $X = [-2, 2) \cup \set{3}$ with the subspace topology inherited from $\R$, let $E = (-1, 0) \cup (0, 1] \cup \set{-2, 3}$. 
    $$\overline{E} = [-1, 1] \cup \set{-2, 3}, \quad \mathring{E} = (-1, 0) \cup (0, 1) \cup \set{3}$$
    $$\operatorname{isolated} (E) = \set{-2, 3}, \quad \operatorname{Lim} (E) = [-1, 1]$$
    And 
    $$\overline{X \setminus E} = [-2, -1] \cup \set{0} \cup [1, 2), \quad \operatorname{int}({X \setminus E}) = (-2, -1) \cup (1, 2)$$
    $$\operatorname{isolated} (X \setminus E) = \set{0}, \quad \operatorname{lim} (X \setminus E) = [-2, -1] \cup [1, 2)$$
    And the boundary is the same 
    $$\partial E = \partial X \setminus E = \set{-2, -1, 0, 1}$$
\end{example}

\begin{definition}
    A set $D \subseteq X$ is said to be \emph{dense} if and only if
    $$ \overline{D} = E$$
    Similarly, a set $D$ is \emph{dense in $E \subseteq X$} iff 
    $$\overline{D \cap E} \supseteq E$$
\end{definition}

\begin{proposition}
    $D \subseteq X$ is said to be dense iff $\forall$ open $U \subseteq X$, $U \neq \emptyset$, 
    $$U \cap D \neq \emptyset$$
    Furthermore, if $\mathcal{B}$ is a basis for $\tau_X$, then $D \subseteq X$ is dense iff $\forall B \in \mathcal{B}$, $B \neq \emptyset$, 
    $$B \cap D \neq \emptyset$$
    However we don't have a similar characterization in terms of sub-basis (Take the normal sub-basis of $\R$ and compare with $\Z$). 
\end{proposition}

\section{Functions of Topological spaces} 

\subsection{Continuous functions} 
Let $f \colon X \to Y$ be a map between two topological spaces, $X$ and $Y$. Then the following are equivalent: 
\begin{enumerate}
    \item $\forall$ open $U \subseteq Y$, $f^{-1} (U)$ is open. 
    \item $\forall$ closed $V \subseteq Y$, $f^{-1} (V)$ is closed. 
    \item $\forall E \subseteq X$, $f(\overline{E}) \subseteq f(\overline{E})$. 
    \item $f$ is continuous at every point $x \in X$, i.e $\forall B \ni f(x)$ open, $\exists A \ni x$ open such that 
    $$f(A) \subseteq B$$
\end{enumerate}

\begin{proof}
    We will show that $2 \iff 3$: 
    \begin{itemize}
        \item[$\implies$] Let $E \subseteq X$, let $V = \overline{f(E)}$. By $2$, we have that 
        $$f^{-1}(V) \text{ is closed in $X$}$$
        But $f^{-1} (V) \supseteq E$, and since it is closed, by definition of the closure 
        $$f^{-1} (V) \supseteq \overline{E}$$
        Hence, 
        $$f(\overline{E}) \subseteq V = \overline{f(E)}$$

        \item[$\impliedby$] Let $V \subseteq Y$ closed, then $V = \overline{V}$. Now consider 
        $$U := f^{-1}(V) \subseteq X$$
        By $3$, we have that 
        $$f(\overline{U}) \subseteq \overline{f(U)} = \overline{V} = V$$
        Therefore 
        $$\overline{U} \subseteq f^{-1} (V) = U$$
        Hence 
        $$\overline{U} = U \implies \text{$U$ is closed}$$  
    \end{itemize}
\end{proof}

\begin{remark}
    For general topological spaces, the sequential definition of the limit does not hold; we only have 
    $$f \text{ is continuous} \implies \left(\forall x_n \in X, \ x \in \R: \colon x_n \to x \implies f(x_n) \to f(x)\right)$$
    The converse is true for \emph{metrizable} spaces, for huge spaces we might need to generalize to nets instead of sequences. 
\end{remark}

\begin{definition}[Homeomorphism]
    A map $f \colon X \to Y$ is called a \emph{homeomorphism} if 
    \begin{itemize}
        \item $f$ is bijective. 
        \item $f$ and $f^{-1}$ are continuous. 
    \end{itemize}
\end{definition}

\subsection{Hausdorff Spaces} 

\begin{definition}
    A topological space is called \emph{Hausdorff} iff for every $a, b \in X$ $a \neq b$, there exists two neighborhoods $A \ni a, \ B \ni b$ s.t 
    $$A \cap B = \emptyset$$
\end{definition}

\begin{example}
    \begin{itemize}
        \item $\R^d$, in fact every metrizable space is Hausdorff. 
        \item $\set{0,1}^\N$ with the topology generated by cylinders is Hausdorff. Take 
        $$x_1 x_2 \cdots , \ y_1 y_2 \cdots \in \set{0, 1}^\N$$
        Let $n$ be the first index s.t $x_n \neq y_n$, then 
        $$x \in [x_1 x_2 \cdots x_n], \ y \in [y_1 \cdots y_n]$$
        Which are distinct (in fact this space turns out to be metrizable). 
        
        \item Let $X = \set{0, 1}$ with $\tau = \set{\emptyset, \set{0}, \set{0, 1}}$, this is not Hausdorff since we can't separate $0$ and $1$. 
        \item The cofinite topology is not Hausdorff on infinite spaces $X$. 
    \end{itemize}
\end{example}

\paragraph{Properties of Hausdorff spaces: } 
\begin{itemize}
    \item \textbf{Singletons are closed:} Take $y \in \set{x}^c$ then we can take an open neighborhood of $y$ that doesn't contain $x$. i.e 
    $$U \subseteq \set{x}^c \implies \set{x}^c \text{ is open}$$
    Note that we clearly get that all finite sets are closed. 

    \item \textbf{Convergent sequences have exactly $1$ limit:} Suppose we have two distinct limits, then we could separate $x_1, x_2$ by open sets. But 
    by definition of convergence the sequence must be eventually in both neighborhoods, contradiction. 
\end{itemize}

\subsection{Equivalence of topological spaces}

\begin{definition}
    Two spaces $X_1, X_2$ are said to be \emph{topologically equivalent / Homeomorphic} if and only if there exists a homeomorphism $\varphi \colon X_1 \to X_2$.
\end{definition}

\begin{definition}
    An injective continuous map $f \colon X \to Y$ is called an \emph{embedding} if and only if 
    $$f \colon X \to f(X) \text{ is a homeomorphism w.r.t subspace topology}$$
    i.e iff
    $$f^{-1} \colon f(X) \to  X \text{ is also continuous w.r.t subspace topology}$$
\end{definition}

\begin{example} 
    \begin{itemize}
        \item Let $X = (0, 1)$ and $Y = \R$, define $f(x) \colon X \to Y$ by 
        $$f(x) = \operatorname{cotan} (\pi x) \quad \text{draw it}$$
        is a Homeomorphism. 
    
        \item Let $X = Y = \R$ and $f(x) = x^3$, then $f(x)$ is a Homeomorphism.
    \end{itemize}
\end{example}

\begin{remark}
    Continuity of $f^{-1}$ is essential, i.e if $f$ is bijective and continuous it need not be a homeomorphism: Take $f \colon [0,1) \to S^1$ with 
    $$f(x) = (\cos 2 \pi x, \sin 2 \pi x) \quad \text{walk $2 \pi x$ radians positively around $S_1$}$$
    Its inverse is not continuous, since a point very close to zero has value very close to a point close to $1$, more formally 
    $$f^{-1} \text{ is not continuous at } (1, 0)$$
\end{remark}

\begin{example}
    Examples of equivalent spaces:
    \begin{itemize}
        \item $(0, 1) \cong (a, b) \cong (a, \infty) \cong (-\infty, b) \cong \R$ for any $a, b \in \R$.
        \item $D^2 \cong (0, 1) \times (0, 1)$, rescale the ray and the rectangle. 
        \item $\set{0,1}^\N \cong \text{ cantor set in $\R$ with the subspace topology}$, Where 
        $$C = \bigcap_{n = 0}^{\infty} A_n \quad A_0 = [0,1], \ A_{n+1} = A_n \text{ where you remove the middle part of each of the subsets}$$
        The idea is to index all the points in the cantor set using a sequence where $0 \to L$, $1 \to R$, this map would be a Homeomorphism.
        \item Take $S^2 \setminus \set{a} \cong \R^2$ with $a \in S^2$: this is via stereographic projection (3B1B).  
    \end{itemize}
    Non-examples of equivalent spaces: 
    \begin{itemize}
        \item $(0,1) \not\cong [0,1]$ since $[0,1]$ is compact, whereas $(0,1)$ is not compact. 
        \item $\Q$ with subspace topology is not Homeomorphic to $\R$, since they have different cardinalities.
        \item $\set{0,1}^\N \not\cong \R$ $\R$ has no nontrivial clopen sets, whereas every cylinder in $\set{0,1}^\N$ is clopen.
        \item $S^1 \not\cong [0, 1) \text{ nor } [0,1]$. For $[0, 1)$ use compactness, for the $[0,1]$ think of paths. 
        \item $\Z$ with the evenly-spaced topology $\not\cong \set{0,1}^\N$, different cardinalities. 
        \item $S^2 \not\cong S^1 \times S^1$ notice that $T^2 \cong S^1 \times S^1$ (fundamental group).
        \item $\R \not\cong \R^2$, paths again. 
    \end{itemize}
\end{example}

\subsection{Topological Properties / Topological Invariants:} 
A property $P$ is said to be \emph{a topological property} if it is shared among Homeomorphic spaces. I.e 
$$X \cong Y, \ X \text{ has } P \implies Y \text{ has } P$$

\begin{example}
    \begin{itemize}
        \item Cardinality. 
        \item Hausdorff property. 
        \item Connectedness / path connectedness. 
        \item Compactness. 
        \item Metrizability. 
        \item Fundamental group. 
    \end{itemize}
\end{example}

\paragraph{More on continuity: }
\begin{itemize}
    \item All constant functions. 
    \item The inclusion map 
    $$i \colon x \to x$$
    \item Composition of two continuous functions is continuous. 
    \item If $f \colon X \to Y$ is continuous and $X_0 \subseteq X$, then 
    $$f_{| X_0} \text{ is continuous}$$
    \item Projections: $\pi_X \colon X \times Y \to X$ with 
    $$\pi_X (x, y) \to x$$
    is continuous w.r.t to the topology. 
    \item Pasting on open sets / closed sets: Let $A, B \subseteq X$ (be open / closed), $f \colon A \to Y$ and $g \colon B \to Y$ which are continuous 
    w.r.t the subspace topology. Define $h \colon A \cup B \to Y$ by 
    $$h(x) := \begin{cases}
        f(x) \quad \text{ if $x \in A$} \\
        g(x) \quad \text{ if $x \in B$}
    \end{cases}$$
    then 
    $$f_{A \cap B} = g_{A \cap B} \implies h (x) \text{ continuous}$$
\end{itemize}

\begin{proof}[Proof of pasting on opens]
    \mbox{}\\
    Let $V$ be open, then 
    $$h^{-1}(V) = f^{-1}(V) \cup g^{-1} (V)$$
    but $f^{-1} (V)$ and $g^{-1}(V)$ are open in $A, B$ respectively. 
    Therefore $\exists U_A, U_B$ open in $X$ such that 
    $$U_A \cap A = f^{-1}(V), \quad U_B \cap B - g^{-1} (B)$$
    Therefore we can write 
    $$h^{-1} (V) = (U_A \cap A) \cup (U_B \cap B)$$
    But $U_A \cap A, \ U_B \cap B$ are both open in $X$ (finite intersection of open sets), and therefore 
    $$h^{-1} (V) \text{ is open in } X$$
    Therefore $h$ is continuous.  
\end{proof}

\begin{lemma}[Urysohn's lemma]
    Let $X$ be a topological space and let $A, B \subseteq X$ be disjoint closed sets. We want to construct a 
    $f \colon X \to [0,1]$ continuous function such that 
    $$f(A) = \set{0}, \quad f(B) = \set{1}$$      
\end{lemma}

\begin{theorem}[Tietze extension theorem]
    $X, Y$ topological spaces, and $A \subseteq X$ closed. And $f \colon A \to Y$ continuous w.r.t subspace topology. We ask: \\
    is there a continuous function $g \colon X \to Y$ such that 
    $$g_{|A} = f$$
\end{theorem}

When and how can we extend a continuous function from a dense subspace to the entire space? For example
$$f \colon \Q \to \R, \ \Q \subseteq \R, \quad \text{can $f$ be extended continuously to $\R$}$$

\subsection{Infinite product spaces}

\paragraph{Product topology and box topology}
Let 
$X = X_1 \times X_2 \times \cdots = \prod_{i=1}^\infty X_i$
\begin{itemize}
    \item \textbf{Box topology:} Use the basis
    $$\mathcal{B}_{\square}:= \set{U_1 \times U_2 \times \cdots \colon \ U_i \in X_i}$$

    \item \textbf{Product topology} Use the subbasis
    $$\mathcal{S}_{\otimes} := \set{U_1 \times X_2 \cdots} \cup \set{X_1 \times U_2 \times \cdots} \cup \cdots$$ 
\end{itemize} 
Note that the Product topology is coarser than the Box topology. It turns out that the product topology is much more interesting 
than the Box topology (think about the difference in convergence between both). 

Let $X_i \neq \emptyset$ be an arbitrary set, and let $\set{X_i}_{i \in I}$ We denote 
$$\prod_{i \in I} X_i = \set{\text{set of all tuples indexed by $I$ of the form } x = (x_i)_{i \in I} \text{ Where $x_i \in X_i$}}$$
In other words 
$$x \colon \ I \to \bigcup_{i \in I} X_i, \quad i \to x_i \in X_i$$
\begin{example}
    \begin{itemize}
        \item If $I = \N$ and $X_i = \set{0,1}$, then 
        $$\prod_{i \in I} x_i = \set{0,1}^\N$$

        \item $I = \R$ and $X_i = [0,1]$, then 
        $$\prod_{i \in I} X_i = [0, 1]^\R$$
    \end{itemize}
\end{example}

\begin{remark}
    Consider the projection map
    $$\pi_k \colon (x_i)_{i \in I} \to x_k$$
    If $U_k \subseteq X_k$, 
    $$\pi_k^{-1} (U_k) = \set{x \in \prod_{i \in I} x_i \colon x_k \in U_k}$$
    Then the product topology is generated by this subbasis. 
    $$\mathcal{S}_{\otimes} = \bigcup_{i \in I} \pi_i^{-1} (U_i) \quad U_i \text{ open in } X_i$$
    Indeed, note that 
    $\mathcal{B} (\mathcal{S}_{\otimes})$ consists of sets of the form 
    \begin{align*}
        B &= \pi_{k_1}^{-1} \cap \cdots \cap \pi^{-1}_{k_m} (U_{k_m}) \\
        &= \prod_{j = 1}^m U_{k,j} \times \prod_{i \in I \setminus \set{k_1, \cdots, k_m}} X_i 
    \end{align*}

    \smallskip\noindent In other words, the product topology is the \emph{coarsest} topology that makes the projection maps continuous. 
\end{remark}

This is a nice way to generate topologies, i.e if we have
$$X \overrightarrow{f_i} X_i \quad \text{weak topology}$$
Then this induces the \emph{coarsest} topology on $X$ that makes the $f_i$s continuous. \\
If we have a family of topological spaces 
$$X_i \overrightarrow{f_i} X \quad \text{strong topology}$$
Then this induces the \emph{finest} topology on $X$ that makes all the $f_i$ continuous.  

The product topology helps us characterize the product space. 
\begin{itemize}
    \item All $X_i$ Hausdorff $\implies$ $\prod_{i \in I} X_i$ is also Hausdorff
    \item All $X_i$ metrizable, and $I$ countable $\implies$ $\prod_{i \in I} X_i$ is metrizable. 
    \item All $X_i$ are compact $\implies$ $\prod_{i \in I} X_i$ is also compact (Tychonov's theorem). 
    \item All $X_i$ are connected / path-connected $\implies$ $\prod_{i \in I} X_i$ connected. 
\end{itemize}

\begin{example}
    $X = \set{0, 1}^\N$, let $\tau_0 = \text{box topology}$ $\tau_1 = \text{topology inherited by the cylinders}$, $\tau_2 = \text{product topology on $\set{0,1}$-discretely topologized}$. 
    Notice that $\tau_0$ is really boring, just the discrete topology. Whereas we have 
    $$\tau_1 = \tau_2$$
    since 
    $$[w] = \bigcap_{i =1 }^n \pi_i^{-1} (\set{w_i}) \implies \tau_1 \subseteq \tau_2$$
    and 
    $$\pi^{-1} (\set{a}) = \bigcup_{w \in \set{0,1}^{k-1}} [wa] \implies \tau_2 \subseteq \tau_1$$
\end{example}

\paragraph{Convergence in $\prod_{i \in I} X_i$.} 
\begin{proposition}
    Let $x^{(1)}, x^{(2)}, \cdots$ a sequence in $\prod_{i \in I} X_i$ and $x \in \prod_{i \in I}$. 
    $$x^{(n)} \to x \iff \forall i \in I, \ x_i^{(n)} \to x_i$$
\end{proposition}
This is because for all $U \ni x \in S_{\otimes}$ $\exists n_0 \ \forall n \geq n_0, \ x^{(n)} \in U$ which is equivalent to $\forall k \in I$ and $\forall U_k$ s.t $x \in \pi^{-1}(U_k)$, 
by definition we have $x_k \in U_k$ and hence $\exists n_0$, $\forall n \geq n_0$ we have 
$$x^{(n)}_k \in U_k \iff x^{(n)} \in \pi_k^{-1} (U_k)$$

Whereas if we take $[0,1]^\N$ with box topology, and 
$$x^{(n)} = (1/n, 1/n, \cdots)$$
Then $x^{(n)}$ does not converge: let $U_k = [0, 2^{-k})$ and take 
$$U = \prod_{k \in \N} U_k$$

\paragraph{Continuity of $f \colon \ X \to \prod_{j \in J} Y_j$.} 
\begin{proposition}
    Let $f \colon \ X \to \prod_{j \in J} Y_j$ is continuous iff 
    $$\forall j \ \pi_j \circ f \text{ is continuous}$$
    Equivalently $\forall j$ $\exists f_j \colon x \to y_j$ continuous s.t 
    $$f(x) = (f_j(x))_{j \in J}$$  
\end{proposition}

\paragraph{Continuity of $f \colon \prod_{i \in I} X_i \to Y$.} There is no nice characterization here. 


\subsection{Metrizable spaces} 
\begin{definition}
    A \emph{metric} on a set $X$ is just a function $d \colon \ X \times X \to \R$ such that 
    \begin{enumerate}
        \item $d(x, y) \geq 0$ with $d(x, y) = 0 \iff x = y$. 
        \item $d(x, y) = d(y, x)$ 
        \item $d(x, y) \leq d(x, z) + d(z, y)$ $\forall x, y, z \in X$.  
    \end{enumerate} 
\end{definition}

\paragraph{Topology associated with a metric. } This is the topology generated by the basis of open balls
$$B_r(x) = \set{y \in X \mid \ d(x, y) < r} \quad x \in X, \ r \in \R$$
Note basis: for union it is trivial, for intersection of $B_r(x), B_s(y)$ take 
$$t \leq r - d(x, z) \xrightarrow{\Delta\text{-ineq}} B_t(z) \subseteq B_r(x)$$
Then take 
$$\min \set{r - d(x, z), s - d(y,z)} \subseteq B_r(x) \cap B_s (y)$$

\begin{example}
    On $R^n$, $x = (x_1, \cdots, x_n), \ y = (y_1, \cdots, y_n)$ 
    \begin{itemize}
        \item Euclidean metric. 
        $$d_2(x, y) = \sqrt{\sum_i (y_i - x_i)^2}$$
        \item Taxicab metric 
        $$d_1(x, y) = \sum_i |y_i - x_i|$$
        
        $$d_\infty = \max_i |y_i - x_i|$$
    \end{itemize}    
    Are these topologically equivalent + (draw them)?  
\end{example}

\begin{definition}
    We say that two metrics $d_1, d_2$ on a set $X$ are \emph{topologically equivalent}, if they generate the same topology. 
\end{definition}

\begin{remark}
    Every metric $d$ is equivalent to a bounded metric. Indeed, 
    $$\overline{d} (x, y) := \min \set{d(x, y), 1} \leq 1$$
    This is basically saying that we are taking the open balls with $r < 1$. 
    Another option 
    $$\tilde{d} := \frac{d(x,y)}{1 + d(x, y)} < 1$$ 
\end{remark}

\begin{remark}
    NOTE: Metric space $\neq$ Metrizable space. 
\end{remark}
 
\begin{example}
    \begin{itemize}
        \item The discrete topology on any set $X$ is always metrizable: 
        $$d(x, y) = \begin{cases}
            1 \quad &\text{if } x \neq y \\
            0 \quad &\text{if } x = y
        \end{cases}$$
        
        \item $X = \set{0, 1}^\N$ with the product topology is metrizable; one metric is
        $$d (x, y) := \begin{cases}
            2^{- \# (x, y)} \quad &\text{if } x \neq y \\
            0 \quad &\text{if } x = y
        \end{cases}$$
        Where 
        $$\#(x, y) := \min \set{n \mid \ x_n \neq y_n}$$
        Another idea is to do 
        $$\tilde{d} (x, y) = \sum_{n = 1}^{\infty} 2^{- n} \Delta_0(x_0,y_0)$$

        \item \textbf{A non-metrizable space:} $X = \set{0, 1}^{I}$ with the product topology where $I$ is uncountable. 
        \begin{proof}
            Suppose that $X$ has a metric $d$, note that for every point
            $$\bigcap_{n \in \N} B_{2^{-n}} (x) = \set{x}$$
            A basis for the product topology on $X$ is generated by the subbasis $\pi^{-1}_k (U)$ where $U$ open in $\set{0, 1}$ and $k \in I$; 
            i.e 
            $$\pi^{-1}_k (U) \quad U \in \set{\set{0}, \set{1}}$$
            Therefore any element of the basis generated by these is of the form 
            $$C(K, w) = \set{x \in X, \colon \ \forall i \in K, \ x_i = w_i} \quad \text{with } K \subseteq I \text{ finite}, \ w \colon \ K \to \set{0,1}$$
            Fix some $x_0 \in X$, consider $B_\varepsilon(x_0)$ we should have some 
            $$C(K, w) \subseteq B_\varepsilon(x_0)$$
            In particular, $\forall n$ $\exists K_n \subseteq I$ finite, and $x \in C(K_n, x_{|K_n}) \subseteq B_{2^{-n}}(x)$. But this implies that 
            $$\bigcap_{n = 1}^{\infty} C(K_n, x_{|K_n}) = \set{x}$$
            Since $\bigcup_{n \in \N} K_n$ is countable and $I$ uncountable, 
            $$\exists k \in I \setminus \bigcup_n K_n$$
            We define $y$ as follows, 
            $$y_i := \begin{cases}
                x_i \quad &\text{if } i \in \bigcup_n K_n \\
                1 - x_i \quad &\text{o.w} 
            \end{cases}$$
            Therefore $y \neq x$, but $\forall K_n$, 
            $$y_ n \in C(K_n, x_{|K_n})$$
            Therefore 
            $$y \in \bigcap_n C(K_n, x_{|K_n})$$
        \end{proof}
    \end{itemize}   
\end{example}

\begin{example}
    Various topologies on Let $X = \R^\N$ (all sequences of real numbers). 
    \begin{itemize}
        \item Product topology, is metrizable
        $$d(x, y) := \sum_{i = 1}^{\infty} \frac{\overline{d}(x_i, y_i)}{2^n}$$

        \item Uniform Topology: defined by 
        $$d(x, y) = \sup \set{\overline{d}(x_i, y_i) \mid \ i \in \N}$$

        \item Box topology: Which is not metrizable. 
    \end{itemize}
    Indeed, 
    $$\text{prod. top. } \subseteq \text{ uniform top. } \subseteq \text{ box top.}$$
\end{example}

\begin{theorem}
    Let $(X_i)_{i \in I}$ be a family of non-trivial metrizable spaces, then 
    $$\prod_{i \in I} \text{ is metrizable } \iff \text{ $I$ is countable}$$
\end{theorem}

\begin{proof}
    \begin{itemize}
        \item[$\impliedby$] this looks like what we did for product topology of $\R^\N$. 
        \item[$\implies$] By contradiction, like what we did for $\set{0, 1}^I$ is not metrizable.
    \end{itemize}
\end{proof}

In a metrizable space, all topological notions can be expressed in terms of convergence of sequence. Indeed 
$$\overline{E} = \set{x \in X \mid \ \exists a_1, \cdots \in E \text{ s.t } x_n \to x}$$
$$\subseteq \text{ is true in general}$$
$$\operatorname{Lim} E = \set{x \in X \mid \ \exists x_1, x_2, \cdots \in E \setminus \set{x} \text{ s.t } x_n \to x}$$
Moreover, $f \colon X \to Y$ with $X$ metrizable: 
$$f \text{ continuous} \iff (x_n \to x) \implies (f(x_n ) \to f(x)), \quad \implies \text{is true in general}$$

\begin{proof}
    Let $X, Y$ be top spaces, and $X$ metrizable, let $f \colon X \to Y$. Then $f$ continuous iff whenever $x_n \to x$ $\implies$ $f(x_n) \to f(x)$. \\
    \begin{itemize}
        \item[$\implies$] This is true in general. 
        \item[$\impliedby$] Suppose that $f$ is not continuous. $\exists V \ni f(x)$ s.t $\forall N \ni x$ we have that 
        $$f(U) \not\subseteq V$$
        Let $d$ be a metric on $X$, then $\forall n$ $f(B_{2^{-n}}) (x) \not\subseteq V$ take 
        $$x_n \in B_{2^{-n}}(x) \ \text{ s.t } f(x_n) \not\in V$$
        But $x_n \to x$ and $f(x_n) \not\to f(x)$. 
    \end{itemize}
\end{proof}

\paragraph{Countability properties} 
\paragraph{First countability. } 
\begin{definition}
    A \emph{neighborhood basis} (local basis) around $x \in X$ is a family $\set{N_i \mid \ i \in I}$ of neighborhoods of $x$ s.t $\forall$ open $U \ni x$, 
    $\exists i$ s.t 
    $$x \in N_i \subseteq U$$
\end{definition}

\begin{definition}
    A space $X$ is said to be first countable if around each point $x$, $\exists$ a countable neighborhood basis. 
\end{definition}

\paragraph{Uniform continuity} 
\begin{definition}
    Let $X, Y$ be metric spaces, a map $f \colon \ X \to Y$ is \emph{uniformly continuous} if
    $\forall \varepsilon > 0, \ \exists \delta > 0$ s.t $\forall x, x' \in X$: 
    $$d(x, x') < \delta \implies d(f(x), f(x')) < \varepsilon$$ 
\end{definition}

\begin{example}[(non-uniformly) continuous function]
    $f \colon (0, 1) \to \R$, $f(x) = \frac{1}{x}$. Let $\varepsilon = 1$, then $\forall \delta > 0$. Let $x = \delta$ and $x' < \delta$ s.t 
    $$\frac{1}{x'} > \frac{1}{x} + 1$$
    Then we have that 
    $$d(x', x) < \delta, \quad d(f(x), f(x')) > \varepsilon$$
\end{example}

\paragraph{Convergence of functions} 
Let $X$ be a set and $Y$ a topological space. We have at least two notions of convergence of maps from $X \to Y$ ($\prod_{i \in X} Y$).
\begin{enumerate}
    \item Pointwise convergence: 
    $$f_n \to f \quad \iff \forall x \in X, \ f_n(x) \to f(x)$$ 
    This is precisely convergence in the product topology! 

    \item Uniform convergence (if $Y$ is metric) 
    $$f_n \xrightarrow{u} f \iff \forall \varepsilon > 0, \ \exists n_0 \ \forall n \geq n_0, \ \forall x \in X, \ |f_n(x) - f(x)| < \varepsilon$$
    This is the notion of convergence w.r.t to the uniform metric on $Y^X$. 
    $$\rho(f, g) := \sup \set{d(f(x), g(x)) \colon \ x \in X}$$
\end{enumerate} 

\begin{example}[sequnce of function that converges pointwise but not uniformly]
    Let $X = Y = [0,1]$ and $f_n(x) = x^n$. Then 
    $$f(x) = \begin{cases}
        1 \quad \text{if } x = 1 \\
        0 \quad \text{o.w}
    \end{cases}$$ 
\end{example}

\begin{theorem}[Uniform Convergence Theorem]
    \mbox{}\\
    Let $X$ be a top. space and $Y$ a metric space. Let $f_n \colon X \to Y$ be a sequence in $Y^X$. If 
    $f_n \to f$ \emph{uniformly}, then $f$ is continuous.     
\end{theorem}
\emph{Interpretation:} $C(X, Y)$ (set of continuous functions from $X$ to $Y$) as a subset $Y^X$ is closed w.r.t uniform topology.\\ 
\emph{Interpretation:} Let $f \colon X \to Y$, if $\forall \varepsilon >0$, $\exists$ continuous $f_\varepsilon \colon X \to Y$ s.t 
$$\rho(f, f_\varepsilon) < \varepsilon$$
then $f$ is continuous. 

\begin{proof}
    Let $x_0 \in X$ and $y_0 \in f(x_0)$. Let $\varepsilon > 0$, then $\exists N$ s.t $\forall n \geq N$ $\forall x \in X$ 
    $$d(f_n(x), f(x)) < \varepsilon$$
    Since $f_N$ is continuous, we have that $\exists N \ni x_0$ s.t $f_N (N) \subseteq B_\varepsilon(f_N(x))$. Hence for every $x \in N$ we have that 
    $$d(f_n(x), f_n(x_0)) < \varepsilon$$
    and 
    $$d (f_n(x), f(x)) < \varepsilon$$
    $$d (f_n(x_0), f(x_0)) < \varepsilon$$
    Then by $\Delta$-ineq 
    $$d(y_0, f(x)) < 3 \varepsilon$$
    Therefore 
    $$f(x) \in B_\varepsilon(y_0) \implies f(N) B_\varepsilon (f(x_0))$$
\end{proof}

\section{Gluing constructions}

\begin{example}
    \begin{itemize}
        \item  We can get $S^1$ by gluing the endpoints of $[0, 2\pi]$. 
        \item We can get the unit sphere $S^2$ by gluing the boundary of the (closed) unit disk in $R^2$. 
        \item Let $X$ be the union of 3 line segments. Identify one of the endpoints, look at the space you get (picture). 
        \item The torus can be constructed from a square via the usual gluing, but also $T = S^1 \times S^1$. 
        \item Start with a closed rectangle, Identify two opposite edges in opposite directions, you get a mobius strip. 
        \item Take the rectangle, identify two opposite edges with the same direction, and two other in opposite directions, you get the Klein bottle. 
    \end{itemize}
\end{example}

\subsection{Two points of view} 
We can think of these gluing in two ways 
\begin{enumerate}
    \item Via a \emph{quotient map}: $q \colon X \to X'$ surjective map with some property, see below. 
    Then we declare $B \subseteq Z'$ open iff 
    $$q^{-1} (B) \text{ is open}$$
    \item Via an \emph{identification relation}: $a \sim b$, note $\sim$ must be an equivalence relation.Then 
    $$X' = X / \sim $$
    We declare $B \subseteq X / \sim$ to be open if 
    $$\bigcup_{[x] \in B} [x] \text{ is open in $X$}$$
\end{enumerate}

\begin{definition}
    Let $Y$ be a topological space, and $f \colon X \to Y$ a map we define the \emph{weak topology on $X$} is the coarsest topology that makes $f$ continuous. 
    Conversely, if $X$ is topological space the \emph{strong topology on $Y$} is the finest topology on $Y$ that makes $f$ continuous. 
\end{definition}

\begin{remark}
    Let $f \colon X \to Y$ and $X$ topological, then the strong topology on $Y$ is 
    $$\tau_Y = \set{B \subseteq Y \mid \ f^{-1} (B) \text{ open in } X}$$
    If $f$ is surjective, then the strong topology is also called the quotient topology. 
\end{remark}

\begin{definition}
    A surjective map $q \colon X \to Y$ is called a \emph{quotient map} if $\forall B \subset Y$, 
    $$B \text{ open } \iff f^{-1}(B) \text{ open}$$
\end{definition}

\begin{definition}
    If $X$ is a topological space, and $f \colon X \to Y$ a surjective map, then 
    $Y$ together with the quotient topology is called a quotient space. 
\end{definition}

\begin{definition}
    If $X$ is a topological space, and $\sim \subset X \times X$ an equivalence relation. The quotient topology on $X / \sim$ is defined as 
    $$B \subseteq X / \sim \text{ open } \iff \bigcup_{[x] \in B} [x] \text{ open in $X$}$$
    We define the quotient map $q \colon X \to X / \sim$ by 
    $$q(x) := [x]$$
\end{definition}

\begin{remark}
    The quotient of a Hausdorff space need not be Hausdorff. Consider $X = [0,1] \times \set{0,1}$ as the disjoint union of $[0,1]$ and $[0,1]$ let 
    $$(x, a) \sim (y, b) \iff (x = y \text{ and } a \neq b \text{ and } x \neq 0) \text{ or } (x = y = 0 \text{ and } a = b)$$
\end{remark}

\begin{proposition}
    Let $X \xrightarrow{q} Y$ be a quotient map, and $Y \xrightarrow{f} Z$. Then 
    $$f \text{ continuous } \iff f \circ q \text{ is continuous}$$
    Draw the picture. 
\end{proposition}

\begin{proof}
    Let $C \subset Z$ be an open set, then $(f \circ q)^{-1} (C) = q^{-1} (f^{-1} (C))$ is open in $X$. 
    But $q$ is a quotient map so this is the case iff $f^{-1}(C)$ is open in $Y$. 
\end{proof}

\paragraph{Question} Let $h \colon X \to Z$ a cont map, does there exist a map $f \colon Y \to Z$ s.t $h = f \circ q$? is it unique?

\subsection{Connected \& disconnected spaces} 
\begin{definition}
    A space $X$ is \emph{disconnected} if $\exists$ non-empty open $A, B$ s.t 
    $$A \cap B = \emptyset, \quad A \cup B = X$$
\end{definition} 

\begin{remark}
    $A$ and $B$ are clopen sets in the definition above, hence 
    $$\text{$X$ is connected } \iff \text{no nontrivial clopen subsets}$$
\end{remark}

\begin{example}
    Connected spaces
    \begin{itemize}
        \item $\R^d$, $[0,1]$, $(0, \infty)$. 
        \item $\R^2$ with lexicographic order topology.
        \item Any ball in $\R^d$.  
        \item Unit disk, remove one diameter, except the center.  
    \end{itemize}
    Disconnected spaces
    \begin{itemize}
        \item $(0,1) \cup (1,2)$ 
        \item $\set{0,1}^\N$, in fact this is \emph{totally disconnected} (i.e the only nonempty connected subsets are singletons). 
        \item $\Q$ is totally disconnected. 
        \item Every nontrivial discrete space. 
        \item Evenly spaced topology of $\Z$. 
    \end{itemize}
\end{example}

\begin{proposition}
    The continuous image of any connected space is connected. 
\end{proposition}

\begin{proposition}
    Let $X$ be any space, the union of any collection of connected subsets of $X$ that have a common element is connected.   
\end{proposition}

\begin{proof}
    Let $\set{E_i}_{i \in I}$ be connected subsets of $X$ s.t $x_0 \in \bigcap_{i \in I} E_i \neq \emptyset$, wlog  
    $$X = \bigcup_{i \in I} E_i$$
    Suppose disconnected, wlog $x \in A$, let $y \in B$. Let $k$ be such that
    $$y \in E_k$$
    Then 
    $$E_k \cap A, \quad E_k \cap B$$
    form a nontrivial open partition of $E_k$, but $E_k$ connected: contradiction. 
\end{proof}

\begin{definition}
    Two sets $A, B$ are \emph{mutually separated} if 
    $$\overline{A} \cap B = A \cap \overline{B} = \emptyset$$
\end{definition}

\begin{proposition}
    If $A, B \subseteq X$ are mutually separated, and $A \cup B$ is $X$, then 
    $$A, B \text{ clopen}$$
\end{proposition}

\begin{proposition}
    Let $X$ be a topological space, then $Y \subseteq X$ is disconnected iff it can be partitioned into two nonempty, mutually separated sets $A, B \subseteq X$ s.t 
    $$Y \subseteq A \cup B \text{ and } A \cap Y \neq \emptyset, \ B \cap Y \neq \emptyset$$
\end{proposition}

\begin{proposition}
    Let $A, B \subseteq X$ be two mutually separated subsets, let $E$ be a connected subset of $X$, and $E \subset A \cup B$, then 
    $$E \subseteq A, \text{ or } E \subseteq B$$ 
\end{proposition}

\begin{proposition}
    If $E \subseteq X$ is connected and $F \supset E \cup \partial E$ ($E \subset F \subset \overline{E}$), then $F$ is connected. 
\end{proposition}

\begin{proof}
    Let $A, B \subseteq X$ be mutually separated sets with $F \subseteq A \cup B$ and 
    $$F \cap A \neq \emptyset$$
    Then $E \subset A \cup B$, hence $E \subset A$ or $E \subset B$. 
    wlog $E \subset A$, hence $\overline{E} \subset \overline{A}$ therefore 
    $$\overline{E} \cap B \subset \overline{A} \cap B = \emptyset$$ 
    Therefore $F \cap B \subset \overline{E} \cap B = \emptyset$. 
    Hence $F = A$, $F$ connected. 
\end{proof}

\begin{theorem}
    The product of a family of connected spaces is connected. i.e 
    $$X_i \ (i \in I) \text{ connected} \iff \prod_{i \in I} X_i \text{ connected}$$
\end{theorem}

\begin{proof}[For finite products]
    Note, this is for product of two, for finite product, just do induction. 

    Let $X,Y$ be connected spaces, suppose that the product is not connected. Let $A, B$ be nonempty disjoint subsets open subsets of $X \times Y$ s.t 
    $$X \times Y = A \cup B$$
    \textbf{Claim 1:} Either $\exists x_0 \in X$ s.t $\set{x_0} \times Y$ intersects both $A$ and $B$, or $\exists y_0 \in Y$ s.t 
    $X \times \set{y_0}$ intersects both $A$ and $B$. Wlog suppose we have such a $y_0$. \\
    Let $A_{y_0} = \set{x \in X \colon (x, y) \in A} \neq \emptyset$, and 
    $B_{y_0} = \set{x \in X \colon (x, y_0) \in B} \neq \emptyset$. \\
    \textbf{Claim 2:} $A_{y_0}, B_{y_0}$ are open in $X$. Observe that 
    $$A_{y_0} \cap B_{y_0} = \emptyset, \quad A_{y_0} \cup B_{y_0} = X$$
    Hence $X$ is disconnected, contradiction. 

    \textbf{Proof of Claim 2:} we will only prove that $A_{y_0}$ is open. Let $x \in A_{y_0}$, then 
    $(x, y_0) \in A$, with $A$ open. By definition of the product topology, $\exists U \ni x$ open in $X$ and $V \ni y_0$ open in $Y$ s.t 
    $$(x_0, y_0) \in U \times V \subseteq A$$
    Therefore $x \in U \subseteq A_{y_0}$, hence $A_{y_0}$ is open. 

    \textbf{Proof of claim 1:} Suppose that (a) is not true, then $\forall x \in X$ we have that 
    $$\set{x} \times Y \subseteq A \text{ or } \set{x} \times Y \subseteq B$$
    Let $(x_1, y_1) \in A, (x_2, y_2 \in B)$, consider $(x_1, y_2)$, Since $(x_1, y_1) \in A$, by the above we have 
    $$(x_1, y_2) \in A$$
    Hence $X \times \set{y_2}$ intersects both $A$ and $B$. 
\end{proof}

\begin{lemma}
    With the notation of the theorem below: 
    $$\bigcup_{J \subset I \text{ finite}} \tilde{X}_J \text{ is dense in } X$$
\end{lemma}

\begin{proof}
    Let $B$ be a basis element for the product topology on $X$, therefore 
    $$B = \bigcap_{k \in J} \pi^{-1}_k (U_k) \quad \text{for some finite $J$, $U_k$ open in $X_k$}$$
    We want to show that 
    $$B \cap \tilde{X}_J \neq \emptyset$$
    Indeed, let $b \in B$ define 
    $$x_i := \begin{cases}
        b_i \quad \text{if} i \in J \\
        a_i \quad \text{if} i \not\in J
    \end{cases}$$
    Then $x \in B$ and $x \in \tilde{X}_j$. 
\end{proof}

\begin{proof}[General case]
    \begin{itemize}
        \item \textbf{Case 1:} $\prod_{i \in I} X_i = \emptyset$, trivial. 
        \item \textbf{Case 2:} $\prod_{i \in I} X_i \neq \emptyset$, take $a \in \prod_{i \in I} X_i$ (define $X = \prod_{i \in I} X_i$). 
        For every \emph{finite} subset $J \subset I$, define 
        $$\tilde{X}_J := \set{x \in X \colon \forall i \not\in J \ x_i = a_i}$$
        Notice that 
        $$\tilde{X}_J \cong \prod_{i \in J} X_i \quad \text{connected}$$
        Since $a \in \bigcap_{J \subset I \text{ finite}} \tilde{X_j}$, then 
        $$\bigcup_{J \subset I} \tilde{X_j} \quad \text{is connected}$$
        Therefore by proposition above 
        $$X = \overline{\bigcup_{J \subset I} \tilde{X}_J} \quad \text{ is connected}$$

    \end{itemize}
\end{proof}

\begin{definition}
    Let $X$ be an arbitrary space, we write $x \sim y$ if $\exists$ connected $C \subseteq X$ s.t 
    $$x, y \in C$$
    Note that this is an equivalence relation, we call the equivalence class the \emph{connect components} of $X$
\end{definition}

\begin{proposition}
    Each connected component of a space $X$ is connected. 
\end{proposition}

\begin{proof}
    Let $C$ be a connected component, suppose that we have $A, B \subset X$ mutually separated sets s.t 
    $$A \cap C, B \cap C \neq \emptyset, \quad (A \cup B) \cap C = C$$
    Pick $x \in A \cap C$ and $y \in B \cap C$, since $x \sim y$ 
    $$\exists E \subset X \text{ s.t } \set{x,y} \subset E$$
    Observe that $E \subset C$, then we have that proposition $5$ either $E \subset A$ or $E \subset B$: contradiction. 
\end{proof}

\begin{remark}
    For $x \in X$, the connected component containing $x$ denoted $C(x)$ is the largest connect subset of $X \ni x$. 
\end{remark}

\begin{example}
    \begin{itemize}
        \item The connected components of $(0,1) \cup (1,2)$ are exactly $(0,1), (1,2)$. 
        \item The connected components of $\Q$ are the singletons. 
        \item the connected components of $\set{0,1}^\N$ are the singletons. 
    \end{itemize}
\end{example}

\begin{proposition}
    Each connected component of $X$ is closed in $X$. 
\end{proposition}

\begin{proof}
    Let $C$ be a connected component, by (6), $\overline{C}$ is also connected, hence by maximality of $C$ 
    $$\overline{C} \subset C$$
\end{proof}

\paragraph{Connected sets in $R$ wiht the Euclidean topology} 
\begin{proposition}
The connected subsets of $\R$ are exactly the intervals: a set $I\subset\R$ is connected iff it is an interval.
\end{proposition}

\begin{proof}
\begin{itemize}
\item[$(\Rightarrow)$] 
Contrapositive. If $I$ is not an interval, then there exist $x,y\in I$ and $z\in\R$ with $x<z<y$ and $z\notin I$. Set
\[
U:=I\cap(-\infty,z),\qquad V:=I\cap(z,\infty).
\]
Then $U,V$ are nonempty, disjoint, open in the subspace $I$, and $I=U\cup V$. Hence $I$ is disconnected.

\item[$(\Leftarrow)$] 
Assume $I$ is an interval. Suppose $I=U\cup V$ with $U,V\subset I$ nonempty, disjoint, open in the subspace $I$. Pick $a\in U$ and $b\in V$ with $a<b$ (swap $U,V$ if needed). Since $I$ is an interval and $a,b\in I$ with $a<b$, we have $[a,b]\subset I$.

Let $S:=U\cap[a,b]$, which is nonempty and bounded above by $b$. Let $s:=\sup S$. Then $s\in[a,b]\subset I$.

\smallskip
\emph{Case 1: $s\in U$.} Because $U$ is open in $I$, there exists $\varepsilon>0$ such that 
\[
(s-\varepsilon,s+\varepsilon)\cap I\subset U.
\]
Since $s<b$, choose $t\in (s,\min\{s+\varepsilon,b\})$. Then $t\in I$ and hence $t\in U$, which contradicts that $s$ is an upper bound of $S$.

\smallskip
\emph{Case 2: $s\in V$.} Because $V$ is open in $I$, there exists $\varepsilon>0$ such that 
\[
(s-\varepsilon,s+\varepsilon)\cap I\subset V.
\]
By the definition of supremum, there exists $u\in S$ with $s-\varepsilon<u\le s$. Then $u\in (s-\varepsilon,s+\varepsilon)\cap I\subset V$, but $u\in S\subset U$, contradicting $U\cap V=\varnothing$.
\smallskip
Both cases are impossible, so no such separation exists. Hence $I$ is connected.
\end{itemize}
\end{proof}

\begin{theorem}[The intermediate Value Theorem]
    Let $X$ be a connected space, and $f \colon X \to \R$ continuous. Let $a, b \in X$ be distinct. Then, $f$ takes all the values 
    between $f(a)$ and $f(b)$
\end{theorem}

\begin{proof}
    $X$ is connected, therefore $f(X)$ is connected, hence $f(X)$ interval.
\end{proof}

\subsection*{Path-connectedness} 
\begin{definition}
    A space $X$ is \emph{path connected} if $\forall a, b \in X$ $\exists$ a continuous $\gamma \colon [0,1] \to X$ s.t 
    $$\gamma(0) = a, \gamma(1) = b$$
\end{definition}

\begin{proposition}
    Every path connected space is connected. 
\end{proposition}

\begin{example}[Connected but not path-connected]
    Consider 
    $$X := \set{(x, \sin 1/x) \colon x \in (0, \infty)} \subseteq \R^2$$
    Notice that 
    $$\overline{X} = X \cup \set{(0,y) \colon y \in [-1, 1]}$$
    Claims: 
    \begin{itemize}
        \item $\overline{X}$ is connected, this is because $X$ is the continuous image of a connected space. 
        \item $\overline{X}$ is not path connected. 
    \end{itemize}
    \begin{proof}[proof of claim $2$]
        Suppose that $X$ is path connected, in particular there is a path 
        $$\gamma \colon [0,1] \to \overline{X}$$
        s.t $\gamma(0) = (1, \sin 1)$ and $\gamma (1) = (0,0)$. Let $x(t)$ and $y(t)$ 
        be the coordinates of $\gamma(t)$, note that these are continuous maps 
        with $x \colon [0,1] \to [0, \infty)$, $y \colon [0,1] \to [-1,1]$. 
        Let (note that the set is nonempty since $x(1) = 0$)
        $$a := \inf \set{t \geq 0 \colon x(t) = 0}$$
        By continuity of $x$ in $\R$, we know that 
        $$x(a) = 0$$
        As $t \uparrow a$, we must have that 
        $$x(t) \downarrow 0, \quad  y(t) \to y(a)$$
        \textbf{Claim:} $\forall b \in [-1,1]$, $\exists t_n \uparrow a$ s.t 
        $$y(t_n) \to b$$
        Contradiction to the convergence of $y(t) \to y(a)$.
        
        \textbf{Proof of the claim:} Let $\theta = \arcsin b \in [- \pi / 2, \pi/2]$. Note: for $t < a$, 
        $x(t) > a \implies y(t) = \sin (1/ x(t))$. For each $n$, consider the interval $(a - a/n, a)$. Pick 
        $m_n$ s.t, $0 < \frac{1}{2 \pi m_n} < x (a - a/n)$. Then, by the IVT, $\exists t_n \in (a-a/n, a)$ st 
        $$x(t_n) = \frac{1}{2 \pi m_n + \theta}$$
        Notice that 
        $$y(t_n) = \sin \frac{1}{x(t_n)} = \sin (2 \pi m_n + \theta) = b$$ 
    \end{proof}
\end{example}

\section{Compactness}
\begin{definition}
    Let $X$ be a topological space and let $E \subseteq X$.
     \begin{itemize}
        \item We define a \emph{cover} of $E$ in 
        $X$ to be a family $\mathcal{U} \subset 2^X$ s.t 
        $$\bigcup_{A \in \mathcal{U}}  A \supseteq E$$
        \item If every $A$ in $\mathcal{U}$ is open, we call $\mathcal{U}$ an \emph{open cover}. 
        \item A subcover of $\mathcal{U}$ of $E$ is a subfamily $\mathcal{U}' \subseteq \mathcal{U}$ which is again a cover.  
    \end{itemize}
\end{definition}

\begin{definition}
    A space $X$ is said to be \emph{compact} iff every open cover of $X$ has a finite subcover.
\end{definition}

\begin{example}
    \begin{itemize}
        \item $\R$ is not compact: let 
        $$\mathcal{U} = \set{(-n, n) \subset \R \colon \ n \in \N}$$
        This is an open cover of $\R$ but it has no finite subcover. 

        \item $[0,1]$ is compact. 
        \begin{proof}
            Let $\mathcal{U}$ be an open cover of $[0,1]$. Let 
            $$S := \set{x \in [0,1] \mid \ [0,x] \text{ has a finite subcover of $\mathcal{U}$}}$$
            Note that this set is not empty since $0 \in S$. Furthermore, if $x \in S$ and 
            $y \in [0,x]$
            Then $y \in S$. Therefore $S$ is an interval. Take 
            $$e := \sup S \leq 1$$
            Then $[0,e)$ is a subset of $S$. Observe that $e \in S$, since we only have to add one more open set namely 
            $A^* \in \mathcal{U}$ s.t $A^* \ni e$. But $e$ must be one since o.w we would get 
            $$e + \varepsilon \in A^* \implies e + \varepsilon \in S $$
        \end{proof}

        \item $\set{0,1}^\N, [0,1]^\N, [0,1]^\R$ are compact. 
        \item $S^1, S^n$ are compact
        \item Cantor space is compact. 
    \end{itemize}
\end{example}

\begin{proposition}
    If $X$ is compact and $E \subseteq X$ closed, therefore 
    $$E \text{ is compact}$$
\end{proposition}

\begin{proof}
    Let $\mathcal{U}$ be an open cover of $E$ in $X$. Then
    $$\mathcal{U}' = \mathcal{U} \cup \set{X \setminus E}$$
    is an open cover of $E$     
\end{proof}
Note: the converse is not true in general (trivial topology). 

\begin{proposition}
    Suppose that $X$ is Hausdorff, $E$ compact; then 
    $$E \text{ is closed in $X$}$$
\end{proposition}

\begin{proof}
    Let $x \in X \setminus E$, we show that $\exists U \ni x$ open neighborhood s.t 
    $$U \cap E = \varnothing$$
    Since $X$ is Hausdorff, $\forall y \in E$ we have that $y \neq x$ hence we can find an open neighborhood of $V_y \ni x, \ W_y \ni y$ s.t 
    $$V_y \cap W_y = \varnothing$$
    Now 
    $$S = \set{W_y \colon y \in E} \text{ is an open cover of $E$}$$
    Since $E$ is compact, we have $S$ has a finite subcover: there exists $I \subset E$ finite s.t 
    $$\bigcup_{y \in I} W_I \supseteq E$$
    Let 
    $$W := \bigcup_{y \in I} V_y, \quad x \in W$$
    open (finite intersection of opens) s.t 
    $$W \cap E = \varnothing$$
\end{proof}

\begin{proposition}
    Let $X$ be Hausdorff and $E, F \subset X$ be disjoint and compact. Then $\exists N \supset E$ and $M \supset F$ open sets s.t 
    $$M \cap N = \varnothing$$
\end{proposition}

\begin{proof}
    Use the trick above twice. 
\end{proof}

\begin{proposition}
    Let $f \colon X \to Y$ be continuous, with $X$ compact. Therefore 
    $$f(X) \subseteq Y \text{ is compact}$$
\end{proposition}

\begin{theorem}[Tychonoff]
    The product of any family of compact spaces is compact (with the product topology). I.e let $X_i \ (i \in I)$ be compact spaces, then 
    $$\prod_{i \in I} X_i \text{ is compact}$$
\end{theorem}

\begin{example}
    \begin{itemize}
        \item Any closed hypercube $[0,1]^n \subset \R^n$ is compact. 
        \item $\set{0,1}^\N$ is compact. 
        \item $S^1 \times S^1$ the torus. 
        \item $[0,1]^\R$ is compact. 
    \end{itemize}
\end{example}

\begin{proof}
    For now, we will only prove this in the case of two spaces. Let $X, Y$ be compact spaces we want to show that $X \times Y$ is compact. 
    Let $\mathcal{C}$ be an open cover of $X \times Y$. For any $y \in Y$ define $X_y := X \times \set{y}$.
    Note: 
    $$X_y \cong X \implies X_y \text{ is compact}, \quad \mathcal{C} \supseteq X_y$$
    Therefore $\exists \mathcal{C}^0_y \subset \mathcal{C}$ finite subcover of $X_y$. 
    Let $V_y := \bigcup_{A \subset \mathcal{C}^0_y}$ with $V_y \text{ open}$. \\
    \textbf{Claim:} $\exists B_y \ni y$ open s.t $X_y \subset X \times B_y \subset V_y$. \\
    Assuming the claim holds, we have that 
    $$\mathcal{E} := \set{B_y \colon y \in Y}$$
    is an open cover of $Y$; hence $\mathcal{E}$ has a finite subcover. Hence $\exists Y^0 \subset Y$ s.t 
    $$\mathcal{E}^0 := \set{B_y \colon y \in Y} \text{ covers $Y$}$$
    Therefore 
    $$X \times Y \subseteq \bigcup_{y \in Y^0} X \times B_y \subset \bigcup_{y \in Y^0} V_y \subset \bigcup_{y \in Y^0} \bigcup_{A \in \mathcal{C}^0} A$$
    Hence  
    $$\mathcal{C}^0 := \bigcup_{y \in Y^0} \mathcal{C}^0_y \text{ is a finite subcover}$$

    \textbf{Proof of the claim:} Since $V_y$ is open; $\forall x \in X$ since $(x,y) \in V_y$ then $\exists$ open $I_{x,y} \subset X$ and $J_{x,y} \subset Y$ s.t 
    $$(x,y) \in I_{x,y} \times J_{x,y} \subset V_y$$
    Now 
    $$D_y := \set{I_{x,y} \times J_{x,y} \colon x \in X}$$
    open cover of $X_y$; by compactness $\exists$ a finite subcover of $D_y$ that covers $X_y$. Now you can take 
    $$B_y = \bigcap_{y \text{ in that finite subcover}} J_{x,y}$$
    Hence $B_y$ open with $X \times B_y \subset V_y$ 
\end{proof}

\begin{proposition}
    Let $B$ be a basis for the topology of $X$ and $E \subseteq X$, then 
    $$E \text{ compact iff every $0$-cover of $E$ has a finite-subcover} $$
\end{proposition}

\begin{theorem}[Alexander subbasis theorem]
    Let $S$ be a subbasis for the topology of $X$ and $E \subset X$; then $E$ is compact iff 
    every $S$-cover of $E$ has a finite subcover. 
\end{theorem}

\begin{example}
    \begin{itemize}
        \item $[0,1]$ is compact; with $S = \set{(-\infty, b)} \cup \set{(a, \infty)}$. 
        \item Tychonoff's theorem (general case). Suppose $X_i \ (i \in I)$ are compact sets. Let $X := \prod_{i \in I} X_i$. 
        Recall 
        $$S = \set{\pi^{-1}_i (U) \colon i \in I, U \subset X_i \text{ open}}$$
        Let $C := \set{\pi_i^{-1} (U) \colon i \in I, \ U \in \mathcal{U}_i}$ be an $S$-cover of $X$. we have two cases
        \begin{enumerate}
            \item $\exists i \in I$ s.t $\mathcal{U}_i$ is a cover of $X_i$; trivial. 
            \item $\forall i \in I$, $\exists x_i \in X_i$ s.t 
            $$x_i \not\in \bigcup{U \in \mathcal{U}_i} U$$
            Fix such and $x_i$ for each $i \in I$; let 
            $$x := (x_i)_{i \in I}$$
            Then $x$ is not covered by $C$; contradiction. 
        \end{enumerate}
        Therefore $C$ has a finite $S$-subcover. 
    \end{itemize}
\end{example}

\subsection{Convergence and compactness in terms of sequences. }
For $E \subset \R^n$ the following are all equivalent: 
\begin{itemize}
    \item Every open cover of $E$ has a finite subcover. 
    \item Every sequence in $E$ has a convergent subsequence. 
    \item $E$ is closed and bounded. 
\end{itemize}

\begin{remark}
    For general spaces; $I$ and $II$ are not equivalent. Look at 
    $$\set{0,1}^\R$$
    it is compact; but not sequentially compact. Think about this. 

    Conversely 
    $$\set{x \in \set{0,1}^\R \colon x_i = 0 \text{ for all but countably many $i$'s}}$$
    is sequentially compact but not compact.
\end{remark}
\begin{theorem}
    Let $X$ be a metrizable space. Then $X$ is compact iff $X$ is sequentially compact.
\end{theorem}

\begin{proof}
    Fix a metric $d$ inducing the topology on $X$.

    \begin{itemize}

    %%%%%%%%%%%%%%%  <= direction
    \item[$\impliedby$] 
    Suppose $X$ is sequentially compact and let $\mathcal{U}$ be an open cover of $X$.

    \medskip
    \noindent\textbf{Claim 1:} 
    There exists $\varepsilon>0$ such that for every $x\in X$ there exists 
    $A\in\mathcal{U}$ with
    \[
        B_\varepsilon(x)\subseteq A.
    \]

    \medskip
    \noindent\textbf{Claim 2:}
    For every $\varepsilon>0$, the space $X$ can be covered by finitely many
    $\varepsilon$-balls.

    \medskip
    \noindent\textbf{Proof of Claim 2:}
    Suppose not. Then for some fixed $\varepsilon>0$, $X$ cannot be covered by finitely many
    $\varepsilon$-balls.

    We inductively construct a sequence $(x_n)$ with the property that for
    $i\neq j$,
    \[
        d(x_i,x_j)\ge \varepsilon.
    \]

    Pick any $x_1\in X$.  
    Having chosen $x_1,\ldots,x_{n-1}$, since
    \[
        X\neq \bigcup_{i=1}^{n-1} B_\varepsilon(x_i)
    \]
    by assumption, pick
    \[
        x_n\in X\setminus \bigcup_{i=1}^{n-1}B_\varepsilon(x_i).
    \]

    Then for each $i<j$ we have $d(x_i,x_j)\ge \varepsilon$, so the sequence
    $(x_n)$ has no Cauchy subsequence. In a metric space, every convergent
    subsequence is Cauchy; hence $(x_n)$ has no convergent subsequence. This
    contradicts sequential compactness of $X$. Therefore Claim 2 holds.

    \medskip
    \noindent\textbf{Proof of Claim 1:}
    Suppose not. Then for each $n\in\mathbb{N}$ there exists $x_n\in X$ such that
    \[
        B_{1/n}(x_n) \text{ is not contained in any element of } \mathcal{U}.
    \]

    Since $X$ is sequentially compact, the sequence $(x_n)$ has a convergent
    subsequence $(x_{n_k})$ with limit $x\in X$. Because $\mathcal{U}$ covers
    $X$, pick $A\in\mathcal{U}$ with $x\in A$. Since $A$ is open, choose
    $\varepsilon>0$ with
    \[
        B_\varepsilon(x)\subseteq A.
    \]

    Since $x_{n_k}\to x$, pick $k$ large enough so that
    \[
        d(x_{n_k},x)<\varepsilon/2.
    \]
    Also, since $n_k\to\infty$, pick $k$ so large that
    \[
        1/n_k < \varepsilon/2.
    \]

    For such $k$, any $y\in B_{1/n_k}(x_{n_k})$ satisfies
    \[
        d(y,x)\le d(y,x_{n_k}) + d(x_{n_k},x) < \frac{1}{n_k} + \frac{\varepsilon}{2}
        < \frac{\varepsilon}{2} + \frac{\varepsilon}{2} = \varepsilon.
    \]
    Thus
    \[
        B_{1/n_k}(x_{n_k})\subseteq B_\varepsilon(x)\subseteq A,
    \]
    contradicting the choice of $x_{n_k}$. Hence Claim 1 holds.

    \medskip
    \noindent\textbf{Finish:}
    By Claim 1, choose $\varepsilon>0$ with the property that every
    $\varepsilon$-ball is contained in some $A\in\mathcal{U}$.  
    By Claim 2, finitely many $\varepsilon$-balls cover $X$.  
    Replacing each ball by a corresponding $A\in\mathcal{U}$ still yields a finite
    subcover of $\mathcal{U}$. Therefore $X$ is compact.

    %%%%%%%%%%%%%%%  => direction
    \medskip
    \item[$\implies$]
    Now assume $X$ is compact.  
    Let $(x_n)$ be a sequence in $X$.

    If the set $\{x_n:n\in\mathbb{N}\}$ is finite, some value occurs infinitely
    many times, giving a constant (hence convergent) subsequence.

    If the set is infinite, then since $X$ is compact, every infinite subset has
    a limit point. Let $x$ be such a limit point. Because $X$ is metrizable,
    choose a sequence of neighborhoods $B_{1/k}(x)$. Each contains infinitely
    many terms of $(x_n)$ by definition of limit point. Pick inductively
    \[
        n_1 < n_2 < \cdots
    \]
    with $x_{n_k}\in B_{1/k}(x)$. Then
    \[
        d(x_{n_k},x)<1/k,
    \]
    so $x_{n_k}\to x$. Hence $(x_n)$ has a convergent subsequence.  
    Therefore $X$ is sequentially compact.

    \end{itemize}
\end{proof}


\begin{theorem}
    Moreover for a metric space $X$, $E \subseteq X$
    $$E \text{ compact} \iff E \text{ complete and totally bounded}$$
    where totally bounded = $\forall \varepsilon > 0$, $E$ can be covered by finitely many $\varepsilon$-balls. 
\end{theorem}

\begin{theorem}
    Lebesgue (number) covering lemma; let $X$ be a compact \emph{metric} space and $U$ an open cover. There exists a $\delta > 0$ s.t 
    if $C \subseteq X$ is a set with diameter less than $\delta$, than $C \subseteq A$ for some $A \in U$.
    
    $\delta$ is said to be a Lebsegue number for $U$. 
\end{theorem}

\paragraph{Compactness in terms of closed sets} 
\begin{definition}
    A family $\mathcal{C}$ of subsets of $X$ has the \emph{finite intersection property} (FIP), iff every 
    finite subfamily of $\mathcal{C}$ has nonempty intersection. 
\end{definition}

\begin{proposition}
    Let $X$ be a topological space, then $X$ is compact iff for every family $\mathcal{C} \subseteq 2^X$ of closed sets 
    with the FIP, has nonempty intersection. 
\end{proposition}

\begin{corollary}
    Let $X$ be compact, then the intersection of every nested sequence of nonempty closed sets in $X$ is non-empty 
    $$C_1 \supseteq C_2 \supseteq \cdots$$
\end{corollary}

\begin{theorem}[Compactness theorem] 
    Let $\mathcal{X}$ be a collection of boolean variables, and let $\mathcal{S}$ be a collection of propositions. If every finite subfamily is satisfiable, 
    then $\mathcal{S}$ is satisfiable.     
\end{theorem}

\begin{theorem}[The extreme value theorem]
    Let $X$ be a compact space, and $f \colon X \to \R$ continuous; then $f$ admits a minimum and a maximum. 
\end{theorem}

\begin{proposition}
    Let $f$ be a continuous bijection with $X$ compact $Y$ Hausdorff, then 
    $$f \text{ is a Homeomorphism}$$
\end{proposition}

\begin{example}
    We want to show that $\mathcal{C}$ (the cantor set) is Homeomorphic to $\set{0,1}^\N$. Define 
    $$\varphi \colon \set{0,1}^\N \to \mathcal{C}, \quad \varphi (x_1,x_2,\cdots) = \sum_{n=1}^{\infty} (\frac{2}{3})^n x_n$$
    Note that this is basically a way of "addressing" the points in $\mathcal{C}$ by the "infinite path from root to leaf".  
    By the proposition above; it is enough to show that $\varphi$ is continuous and bijective; exercise.
\end{example}

\begin{proposition}
    Let $X,Y$ metric spaces; $f \colon X \to Y$ continuous; $X$ compact. Therefore 
    $$f \text{ is uniformly continuous}$$
\end{proposition}

\begin{proof}
    Let $\varepsilon > 0$, since $f$ continuous: $\forall z \in X$ $\exists \delta_z > 0$ s.t 
    $$f (B_{\delta_z} (z)) \subseteq B_\varepsilon (f(z))$$
    Note that we have the following open cover of $X$
    $$\mathcal{U} = \set{B_{\delta_z} (z) \colon z \in X}$$
    Let $\delta > 0$ be a Lebesgue of $\mathcal{U}$. Let $x,y \in X$ be s.t 
    $$d(x,y) < \delta$$
    since 
    $$\operatorname{diam} (\set{x,y}) < \delta$$
    $\exists z \in X$ s.t 
    $$\set{x,y} \subseteq B_{\delta_z} (z)$$
    Hence 
    $$d(x,z), d(y,z) < \delta_z$$
    Hence 
    $$d(f(x),f(z)), d(f(y),f(z)) < \varepsilon \implies d(f(x), f(y)) < 2 \varepsilon$$
\end{proof}

\subsection{Local compactness} 
\begin{definition}
    $X$ is said to be \emph{locally compact} if every point $x \in X$ has a compact neighborhood. 
\end{definition}

\textbf{Recall.} $\mathcal{B} \subseteq 2^X$ is a neighborhood basis for $x$ iff for every open $N \ni x$, $\exists B \in \mathcal{B}$ s.t 
$$x \in B \subseteq N$$

\begin{proposition}
    Let $X$ be Hausdorff, then $X$ is locally compact iff every point $x \in X$ has a neighborhood basis consisting of compact sets. 
\end{proposition}

\begin{lemma}
    Let $X$ be Hausdorff and let $A, B \subseteq X$ be disjoint and compact; then $\exists N \supseteq A$ open s.t 
    $$\overline{N} \cap B = \emptyset$$
    
    Especially, $\forall x \in X$ $\forall$ compact $B \not\ni x$, $\exists N \ni x$ open s.t 
    $$\overline{N} \cap B = \emptyset$$
\end{lemma}

\begin{proof}[Proof of the original prop]
    \mbox{}\\
    \begin{itemize}
        \item[$\implies$] Let $X$ be locally compact, $A \ni x$ an open neighborhood; we need to find some compact set $D \subseteq A$. By local compactness; we know that 
        $\exists C \ni x$ compact neighborhood. Wlog $A \subseteq C$ (replace $N$ by $N \cap C$); then 
        $$C \setminus A \text{ closed in $C$}$$
        Since $X$ (and hence $C$) is Hausdorff, we know that $C \setminus N$ is compact with $x \not\in C \setminus N$; therefore by the lemma, $\exists N \ni x$ open s.t 
        $$\overline{N} \cap (C \setminus A) = \emptyset$$
        Let 
        $$D := N \cap \mathring{C}$$
        Notice that 
        \begin{align*}
            x &\in D \\
            D &\text{ is open} \\
            \overline{D} &\text{ is compact} \\
            \overline{D} &\subseteq A
        \end{align*}
        Where the last point is obtained since 
        $$\overline{N} \cap C \cap A^c \implies \overline{N} \cap C \subseteq A$$
        but $\overline{D}:= \overline{N \cap \mathring{C}} \subseteq \overline{N} \cap C \subseteq A$. 
    \end{itemize}    
\end{proof}

%send to soumaille
\begin{proposition}
    Let $X$ be locally compact Hausdorff. 
    \begin{enumerate}[label = \alph*)]
        \item $A \subseteq X$ closed $\implies$ $A$ locally compact 
        \item $B \subseteq X$ open $\implies$ $B$ locally compact 
        \item $A,B \subseteq X$ locally compact $\implies A \cap B$ locally compact. 
    \end{enumerate}
\end{proposition}

\begin{proof}
    \begin{enumerate}[label = \alph*)]
        \item Let $x \in A$ and $K \ni x$ be a compact neighborhood of $x$ in $X$. Then 
        $$A \cap K \text{ is compact with } a \in A \cap \mathring{K}$$

        \item Let $x \in B$ and $K \ni X$ a compact neighborhood in $X$. By the lemma, $\exists$ open $V \ni x$ s.t 
        $$\overline{V} \subseteq B$$
        Since $X$ is locally compact and Hausdorff, $\exists$ compact set $K$ s.t $x \in \mathring{K}$ s.t 
        $$K \subseteq V \subseteq B$$
        Such a $K$ is a compact neighborhood of $x$ in $B$. 
        
        \item Let $x \in A \cap B$, hence $\exists$ compact $K_A \subseteq A, \ K_B \subseteq B$, $\exists$ open $U, V \subseteq X$ s.t 
        $$x \in U \cap A \subseteq K_A, \quad x \in V \cap B \subseteq K_B$$
        Hence $x \in (U \cap A) \cap (V \cap B) \subseteq K_A \cap K_B \subseteq A \cap B$. Hence 
        $K_A \cap K_B$ is a compact neighborhood of $x$ in $A \cap B$.
    \end{enumerate}
\end{proof}

\begin{definition}
    A map $f \colon X \to Y$ is an open map if its maps open sets to open sets. 
\end{definition}

\begin{proposition}
    if $X$ locally compact and $f \colon X \to Y$ continuous and open; then $f(X)$ is locally compact.   
\end{proposition}

\begin{proposition}
    Let $y \in f(X)$, Take $x \in X$ s.t $f(x) = y$. Let $C \ni x$ be a compact neighborhood of $x$, then 
    $$f(C) \text{ compact}$$
    $$f(\mathring{C}) \text{ open}$$
    $$y \in f(\mathring{C}) \subseteq f(C)$$
    So $f(C)$ is a compact neighborhood of $y$.    
\end{proposition}

\subsection{Compactification}
\begin{example}
    $\R$ is not compact, construct $\R^* = \R \cup \set{\infty}$ we still have to define the topology on $\R^*$. 
    Notice that 
    $$\R \cong S^1 \setminus \set{p}$$
    "Just add $p$ back, it will correspond to $\infty$" 
    Then 
    $$\R^* \cong S^1$$
    The same exact reasoning for $\R^n \cong S^n \setminus \set{p}$. 
\end{example} 

\begin{definition}
    A \emph{compactification} of a top. space $X$ is an \emph{embedding} (Homeomorphism onto the image) $\varphi \colon X \to \hat{X}$ into a compact Hausdorff space 
    s.t 
    $$\varphi(X) \text{ is dense in $\hat{X}$}$$
\end{definition}

\paragraph{One point compactification}
$X$ Hausdorff, locally compact, non-compact: 
$$X^* := X \cup \set{\infty}, \quad \varphi^* \colon X \to X^*, \quad \varphi^*(x) = x$$
The topology on $X^*$ 
$$\tau^* = \tau \cup \set{X^* \setminus C \colon C \subseteq X \text{ compact}}$$
verify that this is a valid topology, $X^*$ compact and Hausdorff. 

\begin{theorem}
    Let $X$ be a non-compact, locally compact Hausdorff space. Let $(\hat{X}, \varphi)$ be a compactification of $X$. 
    \begin{enumerate}[label = (\alph*)]
        \item $\exists$ quotient map $h \colon \hat{X} \to X^*$ s.t 
        $$h \circ \varphi = \varphi^*$$
        \[
            \begin{tikzcd}
                X \arrow[r, "\varphi"] \arrow[dr, swap, "\varphi^*"] & \hat{X} \arrow[d, "h"] \\
                & X^*
            \end{tikzcd}
        \]

        \item Furthermore $(\hat{X}, \varphi)$ is a one-point compactification. Then $h$ can be chosen to be a homeomorphism. 
    \end{enumerate}
\end{theorem}

\begin{proof}
    \begin{enumerate}[label = \alph*)]
        \item 
        \[
            h(\hat{x}) =
            \begin{cases}
                \varphi^* \bigl(\varphi^{-1}(\hat{x})\bigr) & \text{if } \hat{x} \in \varphi(X),\\
                \infty & \text{otherwise.}
            \end{cases}
        \]

        \item the $h$ above is a Homeomorphism. 
    \end{enumerate}
\end{proof}

\begin{definition}
    A topological space $X$ is said to be \emph{completely regular} iff 
    \begin{itemize}
        \item $X$ is Hausdorff
        \item For each $p$ and $\forall A \not\ni p$ closed can be separated by some continuous function.\\
        i.e $\exists f \colon \ X \to [0,1]$ s.t 
        $$f(p) = 0, \quad f(x) = 1 \ \forall x \in A$$
    \end{itemize}
    Note: Complete regularity can be proven to be sufficient for the existence of any compactification. 
\end{definition}

\begin{proposition}
    If we are able to embed a topological space $X$ 
    $$\psi \colon X \to [0,1]^J$$
    Define 
    $$\beta X := \overline{\psi(X)} \subseteq [0,1]^J$$
    Then 
    $$(\beta X, \psi) \text{ is a (Stone-\v{C}ech) compactification of $X$}$$
\end{proposition}

\begin{theorem}[A necessary and sufficient condition for the existence of such an embedding]
    \mbox{}\\
    $X$ is \emph{completely regular} iff such an embedding exists. 
\end{theorem}

\begin{proposition}
    Suppose we have 
    $$X \xrightarrow{\psi} \beta X$$
    $$\downarrow f$$
    $$Y$$
    $\exists h \colon \beta X  \to Y$, then $\forall x \in X$ we have that 
    $$f(x) = h(\psi(x))$$
\end{proposition}

\begin{example}
    Some compactifications
    \begin{itemize}
        \item $\R \to \beta \R \cong [-\infty, \infty] \cong [0,1]$
        \item $\N \to \beta \N \cong \set{1/n}_{\N} \cup \set{0} \subseteq [0,1]$
    \end{itemize}
\end{example}

\begin{theorem}
    The countable product of Sequentially compact spaces is sequentially compact. i.e 
    $$X_1, X_2, \cdots \text{ sequentially compact} \implies \prod_{k=1}^\infty$$
\end{theorem} 

\begin{proof}
    Say we have some sequence $(x^{(1)}), (x^{(2)}), \cdots$ be a sequence in $X := \prod_{k=1}^\infty X_k$. 
    \begin{align*}
        (x^{(1)}) &= x_1^{(1)} x_2^{(1)} \cdots\\
        (x^{(2)}) &= x_1^{(2)} x_2^{(2)} \cdots\\
        (x^{(3)}) &= x_2^{(3)} x_2^{(3)} \cdots\\
        &\vdots
    \end{align*}
    Since $X_1$ is sequentially compact, $\exists I_1 \subset \N$ infinite s.t 
    $$x_1^{(n)} \to y_1 \in X_1 \quad \text{ as } n \to \infty \in I_1$$
    Since $X_2$ is compact; chose $I_2 \subseteq I_1$ infinite s.t 
    $$x_2^{(n)} \to y_2 \in X_2 \quad \text{ as } n \to \infty \in I_2$$
    Continue similarly for all $k \in \N$. \\
    This gives us 
    $$I_1 \supseteq I_2 \supseteq \cdots \quad \text{ all infinite}$$
    and 
    $$y = y_1 y_2 \cdots$$
    then for every $k \in I_k$ the $k$'th coordinate converges to $y_k$. Define 
    \begin{align*}
        n_1 &:= \min I_1 \\
        n_2 &:= \min I_2 \setminus \set{n_1} \\ 
        &\vdots
    \end{align*}
    Then at each $k$, for $l > k$ we have that $n_l \in I_l \subseteq I_k$; hence 
    $$x_n^{(k)} \to y_k, \ n \in I$$
    Hence 
    $$(x)^{n} \to y$$
\end{proof}

\section{Separation \& Countability conditions} 
\subsection{Separation}
A space is 
\begin{itemize}
    \item[$T_2$] \textbf{Hausdorff:} iff $\forall x, y \in X$, $x \neq y$ $\exists$ open $A \ni x$, $B \ni y$ s.t 
    $A \cap B = \emptyset$. 

    \item[$T_3$] \textbf{Regular:} iff \begin{enumerate}
        \item $\forall x \in X$ $\forall C \subseteq X$ closed, $x \not\in C$; $\exists$ open $A \ni x$ $B \supseteq C$, $A \cap B \neq \emptyset$. 
        \item Singletons are closed (iff Hausdorff)
    \end{enumerate}

    \item[$T_4$] \textbf{Normal:} iff \begin{enumerate}
        \item $\forall$ closed $C, D \subseteq X$, $C \cap D = \varnothing$, $\exists$ open sets $A \supseteq C$, $B \supseteq D$, $A \cap B = \varnothing$. 
        \item Singletons are closed. 
    \end{enumerate}

    \item[$T_{3.5}$] \textbf{Completely regular} iff \begin{enumerate}
        \item $\forall x \in X$ $\forall$ closed $C \subseteq X$, $x \not\in C$ $\exists$ continuous $f \colon X \to [0,1]$ s.t 
        $$f(x) = 0, f(y) = 1 \quad \forall y \in C$$
        \item Singletons are closed. 
    \end{enumerate}

    \item[$T_1$] $\forall x, y \ x \neq y$ we can find $U \ni x$ open s.t $y \not\in U$
    \item[$T_0$] $\forall x, y \ x \neq y$ we can find $U \ni x$ open s.t $y \not\in U$ or $V \ni y$ open s.t $x \not\in V$.  
\end{itemize}

Note that 
$$T_4 \implies T_{3.5} \implies T_3 \implies T_2 \implies T_1 \implies T_0$$
We get $T_4 \implies T_{3.5}$ by Urysohn's lemma. 

\begin{example}
    \begin{itemize}
        \item $\R$ is normal (nontrivial so far)
        \item Every compact Hausdorff space is normal. 
        \item $(0,1)^\R$ is regular but not normal (needs proof). 
        \item $\R$ with subbasis the Euclidean top + set of rationals is Hausdorff but not regular. 
    \end{itemize}
\end{example}

\subsection{Countability} 
A space is 
\begin{itemize}
    \item \textbf{First countable} if every point has a countable neighborhood basis. 
    \item \textbf{Second countable} if the space has a countable basis. 
    \item \textbf{Separable} if it has a countable dense set. 
\end{itemize}

\begin{example}
    \begin{itemize}
        \item $\R$ is second countable and separable. 
        \item Every metrizable space is first countable. 
        \item $\R / \Z$ is not first countable at $[\Z]$. 
        \item $\set{0,1}^\R$ is not first countable. 
        \item $\R$ with the discrete topology is first countable but not second countable. 
        \item $X$ compact metric $ \implies C(X, \R)$ is separable w.r.t the uniform topology.  
    \end{itemize}
\end{example}

\begin{table}[H]
        \begin{tabular}{c|ccc}
            & Subspaces & Product & Quotient \\
            \hline
            Hausdorff & Yes & Yes & No \\
            Regular & Yes & Yes & No\\ 
            Normal & No & No & No \\
            First Countable & Yes & Only Countable & No\\
            Second Countable & Yes & Only Countable & No\\
            Separable & No & Only continuous & Yes 
        \end{tabular}
\end{table}
Notice that $(0,1)$ is normal whereas $(0,1)^\R$ is not normal and $[0,1]^\R$. \\
Notice that $[0,1]$ with $\tau = \set{\emptyset} \cup \set{A \subseteq X \colon A \ni 0}$ is clearly separable and $(0,1]$ isn't. 

\paragraph{Some Implications} 
\begin{proposition}
    $X$ is metrizable $\implies$ $X$ Is first countable. 
\end{proposition}

\begin{proposition}
    $X$ is metrizable and separable $\implies$ $X$ is second countable 
\end{proposition}

\begin{proposition}
    $X$ second countable $\implies$ $X$ separable. 
\end{proposition}

\begin{proposition}
    $X$ compact and Hausdorff $\implies$ $X$ normal. 
\end{proposition}

\begin{proposition}
    $X$ is metrizable $\implies$ $X$ is normal. 
\end{proposition}

\begin{proof}
    Fix some metric $d$ that generates the topology of $X$. Let $P, Q \subseteq X$ be disjoint closed sets. If $\varnothing \in \set{P,Q}$ this is trivial. \\
    For every $x \in P$ chose $\varepsilon_x$ s.t $B_{\varepsilon_x}(x) \cap Q = \varnothing$; do the same for every $y \in Q$ s.t 
    $B_{\delta_y}(y) \cap P = \varnothing$. Notice that $\forall x \in P, \ \forall y \in Q$ $B_{\varepsilon_x/3} \cap (x) B_{\delta_y / 3} = \varnothing$.  
\end{proof}

\begin{proposition}
    $X$ is regular and second countable $\implies$ $X$ is normal.
\end{proposition}

\begin{proof}
    Let $A, B \subset X$ closed s.t $A \cap B = \varnothing$. Assume $A, B \neq \varnothing$. Let $\beta$ be a countable basis for $X$. 
    By regularity, $\forall x \in X$ $\exists$ open $U_x, W_x$ s.t 
    $$x \in U_x, \quad B \subseteq W_x, \quad U_x \cap W_x = \varnothing$$
    Since $\beta$ is a basis; we can chose 
    $$F_x \in \beta \text{ s.t } x \in F_x \subseteq U_x$$
    Note: $\overline{F_x}$ doesn't intersect $B$. Since $\beta$ is countable clearly 
    $$\set{F_x}_{x \in X} \text{ is countable}$$
    Hence $\exists F_1, F_2, \cdots$ s.t 
    $$\bigcup_{n=1}^\infty F_n \supseteq A,\quad \forall n \ \overline{F_n} \cap B = \varnothing$$
    Similarly, $\exists G_1, G_2, \cdots$ s.t 
    $$\bigcup_{n \in \N} G_n \supseteq B, \quad \forall n \ \overline{G_n} \cap A = 0$$
    Recursively define: \\
    $$F_1' := F_1 \setminus \overline{G_1}, \quad F_n' := F_n \setminus \bigcup_{i=1}^n \overline{G_i}$$
    $$G_1' := G_1 \setminus \overline{F_1}, \quad G_n' := G_n \setminus \bigcup_{i=1}^n \overline{F_i}$$
    Let $\tilde {A} = \bigcup_{n \in \N} F_n'$ and $\tilde{B} := \bigcup{i=1}^n G_n'$; note that they are open, 
    and cover $A$ and $B$ resp. and that 
    $$\tilde A \cap \tilde B = \varnothing$$
    Since if we take $x \in \tilde A \cap \tilde B$ then $\exists n, m, \ n \leq m$ s.t 
    $$x \in F_n' \cap G_m'$$
    THerefore $x\in F_n$, but $x \not\in \overline{F_n}$.    
\end{proof}

\begin{theorem}[Urysonh's Lemma] 
    Let $X$ be normal; let $A, B \subseteq X$ be disjoint closed sets. Then $\exists$ continuous function $f \colon X \to [0,1]$ s.t 
    $f(x) = 0$ for $x \in A$ and $f(x) = 1$ for $x \in B$. 
    
    Note that the converse is also true. 
\end{theorem}

\begin{theorem}[Tietze extension theorem] 
    Let $X$ be normal and $A \subseteq X$ be closed. Every continuous function $f \colon A \to \R$ can be extended to a continuous function on $X$. I.e we can find some $F \colon X \to \R$ s.t 
    $$F_{|a} = f$$
    Moreover if $f(A) \subseteq [a,b]$, then we can choose $F$ so that $F(X) \subseteq [a, b]$       
\end{theorem}

\begin{lemma}
    $X$ is normal iff $\forall$ disjoint closed $A, B \subseteq X$, $\exists$ open $U \subseteq X$, $A \subseteq U$ s.t 
    $$\overline{U} \cap B \neq \varnothing$$
\end{lemma}

\begin{proof}[Proof of Urysohn's lemma]
    \mbox{}\\
    Let $U_{1/2}$ be an open set s.t $A \subseteq U_{1/2}$ and $\overline{U_{1/2}} \cap B = \varnothing$. Let $U_{1/4}, \ U_{3/4}$ be open sets s.t 
    $$A \subseteq U_{1/4}, \ \overline{U_{1/4}} \subseteq U_{1/2}$$
    $$\overline{U_{1/2}} \subseteq U_{3/4}, \ \overline{U_{3/4}} \cap B = \varnothing$$
    draw the picture, this makes sense; continue Recursively. \\    
    We can find $U_{k/2^n}$ open s.t 
    $$A \subseteq U_{1/2^n} \subset \overline{U_{1/2^n}} \subseteq \cdots \subseteq \overline{U_{k/2^n}} \subseteq U_{(k+1)/2^n} \cdots \subseteq \overline{U_{2^n-1/2^n}} \subseteq X \smallsetminus B$$
    The indices are dyadic rations, $o < r< 1$ which are dense in $[0,1]$. \\
    In summary, for each $r \in \Q_2$, $0 < r < 1$ we have some $U_r \subseteq X$ s.t 
    $$\forall r, \ A \subseteq U_r$$
    $$\forall r, \ \overline{U_r} \subseteq X \setminus B$$
    $$r < s \implies \overline{U_r} \subseteq U_s$$
    Define 
    $$f(x) := \begin{cases}
        \operatorname{inf} \set{r \colon x \in U_r} \quad &x \in \bigcup_{0 < r< 1} {U_r} \\
        1 \quad &\text{o.w}
    \end{cases}$$
    Notice that $x \in A \implies f(x) = 0$, $x \in B \implies f(x) = 1$. \\
    Moreover, notice that $x \in U_r \implies f(x) \leq r$; $x \not\in U_r \implies f(x) \implies f(x) \geq r$; $x \in \overline{U_r} \imlies f(x) \leq r$. 
    Hence $\forall s > r$ we have that $x \in U_s \implies f(x) \leq s$. 
    
    Let $x \in X$ and $\varepsilon > 0$: 
    \begin{itemize}
        \item $0 < f(x) < 1$ take $r, s \in \Q_2$ s.t (density)
        $$0 < f(x) - \varepsilon < r < f(x) < s < f(x) + \varepsilon < 1$$
        Note: $x \in U_s \setminus \overline{U_r}$ (observations above); which is an open set. Moreover, $\forall y \in U_s \setminus \overline{U_r}$ we have 
        $r \leq f(y) \geq r$; therefore $f$ is continuous at $x$. 

        \item $f(x) = 0$ do the same. 
        \item $f(x) = 1$ do the same. 
    \end{itemize}
\end{proof}

\subsection{Embedding into cubes, metrization, compactification}
$$\psi \colon X \to [0,1]^J$$
\begin{definition}
    Let $a \in X$ and a closed set $a \not\in B \subseteq X$, $B$ closed. A continuous map $f \colon X \to \R$ separates $a$ from $B$ means that 
    $$f(a) \not\in \overline{B}$$
\end{definition}

\begin{definition}
    A family $\mathcal{F} \subseteq C(X, \R)$ separates points from closed sets if for every $a \in X, B \not\ni a$ closed $\exists f \in \mathcal{F}$ that separates $a$ from $B$.  
\end{definition}

\begin{remark}
    Note: $X$ is completely regular, then $C(X, [0,1])$ separates points from closed sets.  
\end{remark}

\begin{theorem}[Embedding into a cube]
    \mbox{}\\
    Let $X$ be a space in which singletons are closed. Let $\mathcal{F} = \set{f_j \in C(X, [0,1])}_{j \in J}$ be a family of continuous maps that separates points from closed sets. 
    Then
    $$\psi \colon X \to [0,1]^J, \quad \Psi(x) = (f_j(x))_{j \in J}$$  
\end{theorem}

\begin{proposition}
    Suppose that $X$ is regular and second countable. Then there exists a countable family $\mathcal{F} \subset C(X, [0,1])$ that separates points from closed sets. 
\end{proposition}

\begin{corollary}[Urysohn's metrization theorem]
    \mbox{}\\
    Every regular second countable space is metrizable. 
\end{corollary}

\begin{proof}
    We need to show that $\psi$ is a Homeomorphism. 
    \begin{itemize}
        \item \textbf{$\psi$ is one-to-one.} Let $x, y \iN X$, $x \neq y$. Since $\mathcal{F}$ separates points from open sets, it separates points from points. Pick $j$ s.t 
        $f_j$ separates $x$ from $y$, then $\psi(x) \neq \psi(y)$. 

        \item \textbf{$\psi$ is continuous.} $f_j$ is continuous. 
        
        \item \textbf{$\psi^{-1} \colon \psi(X) \to X$ is continuous.} Let $A \subseteq X$ open, NTS that 
        $$(\psi^{-1})^{-1} (A) = \psi(A) \quad \text{open in $\psi(X)$}$$
        Let $w \in \psi(A)$, then $w = \psi(x)$ for some $x \in A$. Since $X \setminus A$ is closed and $a \not\in X \setminus A$. 
        There exists some $j \in J$ s.t $f_j$ separates $x$ from $X \setminus A$. Hence 
        $$f_j(x) \not\in \overline{f_j(X \setminus A)}$$
        Let $\varepsilon > 0$ be s.t $f_j(x) \pm \varepsilon \cap f_j(X \setminus A) = \varnothing$.
        Define the open set
        $$W := \pi_j^{-1}(f_j(x) \pm \varepsilon)$$
        Hence 
        $$W \cap \psi(X) \quad \text{is open in $\psi(X)$}$$
        Moreover, 
        $$W \cap \psi(X) \subseteq \psi(A)$$
        Since 
        $$\psi(y) \in W \implies f_j(y) \in f_j(x) \pm \varepsilon \implies \psi(y) \not\in \psi (X \setminus A)$$
        Moreover 
        $$w \in W \cap \psi(X)$$
    \end{itemize}
\end{proof}

\begin{proof}
    Note: $X$ is normal. Therefore by Urysohn's lemma, $\forall p \in X$ and $\forall$ closed $Q \subset X$, $Q \not\ni p$ we can find 
    $$f \in C(X, [0,1])$$
    that separates $p$ from $Q$. Let $\beta$ be a countable basis, then for each $A, B \in \beta$ with 
    $$\overline{A} \subset B$$ 
    Define 
    $$f_{A,B} \to [0,1], \quad f(\overline{A}) = 0, f(X \setminus B) = 1$$
    and 
    $$J := \set{(A, B) \in \beta^2 \colon \overline{A} \subset B}$$
    \textbf{Claim:} $\mathcal{F} = \set{f_{A,B} \colon (A, B) \in J} \quad \text{separates points from closed sets}$. \\
    Indeed, let $p \in X$ and $Q \not\ni x$ be closed. Since $Q^c$ is open we can find some $B \subseteq Q^c \in \beta$ s.t $B \ni p$. By regularity of $X$ $\exists A \ni p$, $A \in \beta$ 
    s.t 
    $$\overline{A} \subset B$$
    Then 
    $$f_{A, B} \quad \text{separates $p$ from $Q$}$$
\end{proof}

\subsection{Baire spaces}
Notice that if $X$ a topological space and $A_1, A_2,\cdots, A_n$ are dense open sets; then 
$$A_1 \cap A_2 \cap \cdots \cap A_n \quad \text{ dense open}$$

\begin{theorem}[Baire's (CH version)] 
    \mbox{}\\
    Let $X$ be a compact and Hausdorff space, and let $A_1, A_2, \cdots$ be a countable collection of dense open sets in $X$. Then 
    $$\bigcap_{n=1}^\infty A_n \quad \text{is dense in $X$}$$
\end{theorem}

\begin{proof}
    Let $U \subseteq X$ be non-empty and open. Since $A_1$ is dense and open, $\exists x_1 \in U \cap A_1$. Moreover 
    $$A_1 \cap U \ni x_1 \text{ is open}$$
    Since is $X$ is normal (hence regular), 
    $$X \setminus (A_1 \cap U) \not\ni x_1 \quad \text{ is closed}$$
    We can find some open neighborhood $V_1 \ni x_1$ s.t 
    $$\overline{V_1} \subseteq A_1 \cap U$$
    Moreover, since $A_2$ is dense, do the same thing and find some 
    $$x_2 \in A_2 \cap V_1, \ V_2 \ni x_2 \text{ s.t } \overline{V_2} \subseteq A_2 \cap V_1$$
    Repeat this for every $n$; we get some 
    $$V_n \neq \emptyset, \ \overline{V_n} \subset A_n \cap V_{n-1}$$
    This gives us 
    $$V_0 \supseteq \overline{V_1} \supseteq \overline{V_2} \supseteq \cdots, \quad \overline{V_n} \subseteq A_n$$
    By compactness of $X$ we have that 
    $$(\bigcap_{n = 1}^\infty A_n) \cap U \supseteq \bigcap_{n = 1}^\infty V_n \neq \varnothing$$
\end{proof}

\begin{definition}
    A space $X$ is a \emph{Baire space} if the intersection of every countable family of dense open sets is dense. 
\end{definition}

\begin{corollary}
    Compact Hausdorff spaces are Baire. 
\end{corollary}

\begin{theorem}[Baire, LCH]
    Every locally compact Hausdorff space is a Baire space.    
\end{theorem}

\begin{proof}
    Let $X$ be LCH and not compact, let $X^* = X \cup \set{\infty}$ be a one point compactification of $X$. \\
    Note: $X$ is both dense and open in $X^*$. \\
    Let $A_1, A_2, \cdots \subseteq X$ be open and dense: 
    \begin{itemize}
        \item $A_n$ is open in $X$ so it must be open in $X^*$. 
        \item $A_n$ is dense in $X^*$. 
    \end{itemize}
    Therefore 
    $$\bigcap_{n=1}^\infty A_n \quad \text{ is dense in $X^*$}$$
    Hence it is dense in $X$. 
\end{proof}

\begin{lemma}
    Let $X$ be a complete metric space, and let 
    $$C_1 \supseteq C_2 \supseteq C_3 \supseteq \cdots$$
    Be nested nonempty closed subsets of $X$ with 
    $$\operatorname{diam} C_n \to 0 \ \text{ as } n \to \infty$$
    Then 
    $$\bigcap_{n=1}^\infty C_n \neq \varnothing$$ 
\end{lemma}

\begin{proof}
    exercise
\end{proof}

\begin{theorem}
    Every completely metrizable space is Baire. 
\end{theorem}

\begin{definition}
    A set $A \subseteq X$ is \emph{nowhere dense} if it is not dense in any nonempty open set of $X$. 
    i.e iff 
    $$\overline{A} \quad \text{ has empty interior}$$
\end{definition}

\begin{example}
    \begin{itemize}
        \item In $\R$: Finite sets, $\Z$ are nowhere dense. 
        \item In $\R^2$: Finite sets, Lattices, most curves. Plane filling curves are not. 
        \item $A \subseteq X$ open or closed, then $\paratial A$ nowhere dense. 
        \item $A$ is nowhere dense $\implies$ $\overline{A}$ is nowhere dense. 
        \item $A, B$ nowhere dense $\implies$ $A \cap B$ nowhere dense. 
    \end{itemize}
\end{example}

\begin{proposition}
    $A \subseteq X$ is nowhere dense iff $X \setminus A$ has a dense interior. 
\end{proposition}

\begin{proof}
    $\overline{A}$ has empty interior iff $X \setminus \overline{A}$ is dense.  
\end{proof}

\begin{definition}
    \begin{itemize}
        \item A \emph{Meagre set} is a countable union of nowhere dense sets. 
        \item A \emph{Residual set} is a countable intersection of sets with dense interiors. 
    \end{itemize}
    Hence 
    $$A \text{ meagre} \iff X \setminus A \text{ residual}$$1
\end{definition}

\begin{proposition}
    \begin{itemize}
        \item The countable intersection of residual sets is residual. 
        \item The countable union of Meagre sets is Meagre. 
    \end{itemize}
\end{proposition}

\begin{theorem}
    There exists a continuous function $h \colon [0,1] \to \R$ is nowhere differentiable. Moreover, a "generic" $h \in C([0,1], \R)$ has 
    this property. 
\end{theorem}

\begin{proposition}
    $C([0,1], \R)$ with the uniform metric is complete; and hence $C([0,1], \R)$ is Baire. 
\end{proposition}

\begin{proof}
    Let $f_1, f_2, \cdots \in C([0,1], \R)$ be a Cauchy sequence, i.e $\forall \varepsilon > 0, \ \exists n_0, \ \forall n, m \geq n_0$ we have 
    $$d(f_n, f_m) < \varepsilon \implies \forall x \in [0,1], \ |f_n(x) - f_m(x)| < \varepsilon$$
    Therefore for each fixed $x$, $f_n(x)$ is a Cauchy sequence in $\R$ (which is complete); define 
    $$f(x) := \lim_{n \to \infty} f_n(x)$$
    This limit is uniform, hence by the uniform limit theorem 
    $$f \in C([0,1], \R)$$
\end{proof}

\begin{proof}[Proof of the theorem]
    \mbox{}\\
    We construct $D_1, D_2, \cdots \subseteq C([0,1], \R)$ s.t 
    \begin{enumerate}
        \item $\forall n \in \N, \ D_n \text{ is open}$ 
        \item $\forall n \in \N, \ D_n \text{ is dense}$ 
        \item $\forall h \in \bigcap_{n=1}^\infty D_n$ is nowhere differentiable. 
    \end{enumerate}    
    Therefore $\bigcap_{n=1}^\infty D_n$ is residual and consists of nowhere differentiable functions. 
    
    Indeed: 
    $$D_n := \set{f \in C([0,1], \R) \colon \forall x \in [0,1], \ \exists s \in [-1/n,1/n] \setminus \set{0} \text{ s.t } \left|\frac{f(x+s) - f(x)}{s}\right| > n}$$
    Clearly we have that $\bigcap_{n \in \N} D_n$ is nowhere differentiable. Moreover, to show that $D_n$ is open, we will show that $D_n^c$ is closed: Let $f_1, f_2, \cdots \to f$ be a 
    convergent sequence of function in $D_n^c$, we show that $f \in D_n^c$. Since $f_m \not\in D_n$ then $\exists x_m \in [0,1]$ s.t $\forall s \in (-1/n,1/n) \setminus \set{0}$ we have 
    $$\left|\frac{f(x+s) - f(x)}{s}\right| \leq n$$
    Wlog we can assume that $x_m$ converges to $x$, since $[0,1]$ is compact. Exercise, show that 
    $$f_m(x_m) \to f(x)$$
    $$f_m(x_m + s) \to f(x+s)$$
    Hence $\left|\frac{f(x+s) - f(x)}{s}\right| = \lim_{m \to \infty} \left|\frac{f_m(x+s) - f_m(x)}{s}\right| \leq n$. Hence 
    $$f \not\in D_n$$
    To show that $D_n$ is dense, we will show that for any $f \in C([0,1], \R)$ and any $\varepsilon > 0$ there exists some $g \in D_n$ s.t $d(f,g) < \varepsilon$. 
    Take such an $f$: 
    \begin{enumerate}
        \item Pick some polynomial $p$ s.t $d(f, p) < \varepsilon/2$ (exists by Weirestrass approximation theorem). 
        \item Define 
        $$u := \varepsilon/2 \sin \left(\omega_n x \right) \quad \omega_n \text{ to be determined}$$
        Let $g(x) = p(x) + u(x)$ continuous.\\ 
        Notice that 
        $$d(f, g) \leq d(f,p) + d(p, p+u) = d(f, p) + d(0,u) < \varepsilon/2 + \varepsilon/2 = \varepsilon$$
        Let $M := \sup_{x \in [0,1]} |p'(x)|$ (the max slope). Moreover, $\forall x \in [0,1]$, $\exists s \in (-2\pi / \omega_n, 2\pi/\omega_n)$ s.t 
        $$\left|u(x+s) - u(x)\right| \geq \varepsilon/2$$
        Hence by the triangle inequality 
        $$|g(x+s) - g(x)| \geq |u(x+s) - u(x)| - |p(x+s) - p(s)| \geq \varepsilon/2 - s M$$
        Therefore 
        $$\left|\frac{g(x+s) - g(x)}{s}\right| \geq \left|\frac{u(x+s) - u(x)}{s}\right| - M \geq \frac{\varepsilon \omega_n}{4\pi} - M$$
        Hence choose $\omega_n$ s.t 
        $$\frac{\varepsilon \omega_n}{4\pi} - M > n$$
        This gives us that $g \in D_n$
    \end{enumerate}
\end{proof}

\begin{theorem}
    Let $f \colon \R \to \R$ be continuous and $c \in \R$. Suppose that $\forall q > 0$ we have that 
    $$\lim_{n \to \infty} f(nq) = c$$
    Then we get that 
    $$\lim_{x \to \infty} f(x) = c$$ 
\end{theorem}

\begin{proof}
    Fix $\varepsilon>0$ we want to show that $\exists M$ s.t $\forall x \geq M$ we have 
    $$|f(x) - c| \leq \varepsilon$$
    We know that $\forall q > 0$ there exists an $n_q$ s.t $\forall n \geq n_q$ we have that 
    $$|f(nq) - c| \leq \varepsilon$$
    i.e $\forall q > 0$ there exists an $m_q$ s.t $\forall n > m / q$ we have 
    $$|f(nq) - c| \leq \varepsilon$$
    For $m \in \N$, define 
    $$E_m := \set{q > 0 \colon \forall n > m /q \ |f(nq) - c| \leq \varepsilon}$$
    Note: $E_m$ is closed in $(0, \infty)$ since 
    $$E_m = \bigcap_{n \in \N} \set{q > 0, \ q \leq m/n \text{ or } |f(nq) - c| \leq \varepsilon}$$
    By assumption we have that 
    $$\bigcup_{m=1}^\infty E_m = (0, \infty)$$
    Note: $(0, \infty)$ is a Baire space (Homeomorphic to $\R$) and clearly $(0, \infty)$ is open in $(0, \infty)$. 
    Therefore by Baire's theorem, not every $E_m$ is nowhere dense. Hence $\exists m_0$ s.t 
    $$\overline{E_{m_0}} = E_{m_0} \supseteq [q_1, q_2] \ q_2 > q_1 > 0$$
    Hence we have that $\forall q \in (q_1, q_2)$ and $\forall n > m_0/q$ 
    $$|f(nq) - c| \leq \varepsilon$$
    i.e $\forall x = nq > m_0$ for some $q \in (q_1, q_2)$ we have 
    $$|f(x) - c| \leq \varepsilon$$
\end{proof} 

\subsection{Lebesgue covering dimension} 
\begin{definition}
    Let $\mathcal{C}$ be an open cover, the \emph{ply} of $\mathcal{C}$ at $x \in X$ is the cardinality of the set of elements of $\mathcal{C}$ containing $X$. 

    The \emph{ply} of $\mathcal{C}$ is the maximum ply among all $x \in X$. 
\end{definition}

\paragraph{Question.} What is the minimum ply among all covers of $\R^n$ with open balls of radius $1$? 
\begin{itemize}
    \item $n = 1$: We get $2$. 
    \item $n = 2$: We get $3$. 
    \item $\vdots$
\end{itemize}

\begin{definition}
    A \emph{refinement} of an open cover $\mathcal{C}$ is an open cover $\mathcal{D}$ s.t $\forall C \in \mathcal{D}$ we have that $C \in \mathcal{C}$
\end{definition}

\begin{definition}[Lebsegue covering dimension]
    The \emph{covering dimension} of a space $X$ is the smallest $n \in \N$ s.t every cover $\mathcal{C}$ of $X$ has a refinement of ply $n+1$.  
\end{definition}

\begin{example} [Trivial examples]
    \mbox{}\\
    \begin{itemize}
        \item $\operatorname{dim} \varnothing = -1$ 
        \item $\operatorname{dim} \set{x} = 0$ 
    \end{itemize}
\end{example}

\begin{proposition}
    Let $X$ be a compact metric space and $m \in \N$  and suppose that $\forall \varepsilon > 0$, $\exists \mathcal{D}_\varepsilon$ an open cover of $X$ with 
    ply $\leq m+1$ whose elements have diameter $< \varepsilon$.  \\
    Therefore $\operatorname{dim} (X) \leq m$. 
\end{proposition}

\begin{proposition}
    Let $Y$ be a closed subspace of $X$, then $\operatorname{dim} (Y) \leq \operatorname{dim} (X)$. 
\end{proposition}

\begin{proposition}
    $\operatorname{dim} (X \times Y) \leq \operatorname{dim} X + \operatorname{dim} Y$.
\end{proposition}

\begin{theorem}
    Let $X$ be a compact metric space then the following are all equivalent: 
    \begin{enumerate}
        \item $X$ is zero-dimensional 
        \item $X$ has basis consisting of clopen sets
        \item Every distinct pair of points can be separated by clopen sets
        \item $X$ is totally disconnected
    \end{enumerate}
\end{theorem}

\begin{theorem}
    Every compact metric metric space with finite dimension $m$ can be embedded into $\R^{2m+1}$
\end{theorem}

\end{document} 